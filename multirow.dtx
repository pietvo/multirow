% \iffalse meta-comment
%
% Parts of this file, Copyright (C) 1994 by Jerry Leichter
% Copyright (C) 2016-2024 by Pieter van Oostrum <pieter@vanoostrum.org>
% -------------------------------------------------------
%
% This file may be distributed and/or modified under the
% conditions of the LaTeX Project Public License, either version 1.3
% of this license or (at your option) any later version.
% The latest version of this license is in:
%
%    http://www.latex-project.org/lppl.txt
%
% and version 1.3 or later is part of all distributions of LaTeX
% version 2005/12/01 or later.
%
% \fi
%
% \iffalse
%<*driver>
\ProvidesFile{\jobname.dtx}
%</driver>
% This is the variable part of the preamble
%
%<multirow|bigstrut>%% Copyright (C) 1994 by Jerry Leichter
%<bigdelim>%% Copyright (C) 1994 by \O ystein Bache
%% Copyright (C) 2016-2024 by Pieter van Oostrum <pieter@vanoostrum.org>
%%
%
%    \begin{macrocode}
%<multirow|bigstrut|bigdelim>\NeedsTeXFormat{LaTeX2e}
%<multirow>\ProvidesPackage{multirow}%
%<bigstrut>\ProvidesPackage{bigstrut}
%<bigdelim>\ProvidesPackage{bigdelim}
%<multirow|bigstrut|bigdelim>           [2024/11/11 v2.9
%<multirow>                  Span multiple rows of a table]%
%<bigstrut>                  Provide larger struts in tabulars]
%<bigdelim>                  Create big delimiters in tabular or array]
%    \end{macrocode}
%
%<*driver>
\documentclass[a4paper]{ltxdoc}
\usepackage[T1]{fontenc}
\usepackage{multirow,bigstrut,bigdelim,colortbl}
\usepackage[table]{xcolor}
\usepackage{tabulary}
\usepackage{longtable}
\usepackage{booktabs}
\usepackage{calc}
\usepackage{url}
\usepackage[colorlinks, urlcolor=blue, linktocpage]{hyperref}
\usepackage{tikz}
\usepackage{hhline}
\usepackage{etoolbox}
\AtBeginEnvironment{macrocode}{\color{red}}
\newcommand{\minitab}[2][l]{\begin{tabular}{#1}#2\end{tabular}}
\DisableCrossrefs
\CodelineIndex
\RecordChanges
\newcommand{\env}[1]{\texttt{#1}}
\newcommand{\Package}[1]{\textsf{#1}}
\def\bsbs{\cs{\char`\\}}
\begin{document}
  \DeleteShortVerb{\|}
  \DocInput{\jobname.dtx}
\end{document}
%</driver>
% \fi
%
% \CheckSum{0}
%
% \CharacterTable
%  {Upper-case    \A\B\C\D\E\F\G\H\I\J\K\L\M\N\O\P\Q\R\S\T\U\V\W\X\Y\Z
%   Lower-case    \a\b\c\d\e\f\g\h\i\j\k\l\m\n\o\p\q\r\s\t\u\v\w\x\y\z
%   Digits        \0\1\2\3\4\5\6\7\8\9
%   Exclamation   \!     Double quote  \"     Hash (number) \#
%   Dollar        \$     Percent       \%     Ampersand     \&
%   Acute accent  \'     Left paren    \(     Right paren   \)
%   Asterisk      \*     Plus          \+     Comma         \,
%   Minus         \-     Point         \.     Solidus       \/
%   Colon         \:     Semicolon     \;     Less than     \<
%   Equals        \=     Greater than  \>     Question mark \?
%   Commercial at \@     Left bracket  \[     Backslash     \\
%   Right bracket \]     Circumflex    \^     Underscore    \_
%   Grave accent  \`     Left brace    \{     Vertical bar  \|
%   Right brace   \}     Tilde         \~}
%
% \changes{v1.7}{2016/09/13}{Give all the files the same version number}
% \changes{v2.0}{2016/09/27}{Release v2.0}
% \changes{v2.3}{2018/08/03}{Small bugfix}
% \changes{v2.4}{2018/12/30}{Add \leavevmode in bigstrut.sty}
% \changes{v2.4}{2019/01/01}{Make \meta{width} and \meta{vmove} in \cs{multirow}
% \Package{calc} compatible}
% \changes{v2.5}{2019/05/31}{Solve a clash with the \Package{colortbl} package}
%
% \GetFileInfo{multirow.sty}
%
% \DoNotIndex{\#,\$,\%,\&,\@,\\,\{,\},\^,\_,\~,\ }
% \DoNotIndex{\@ne}
% \DoNotIndex{\advance,\begingroup,\catcode,\closein}
% \DoNotIndex{\closeout,\day,\def,\edef,\else,\empty,\endgroup}
%
% \title{The \Package{multirow},
%   \Package{bigstrut} and
%   \Package{bigdelim} packages}
% \author{Pieter van Oostrum\thanks{catalogued ``active author''}\\
%   \O ystein Bache\\
%   Jerry Leichter\thanks{Documentation originally put together by Robin
%     Fairbairns}}
%
% \date{Released \filedate, version \fileversion}
%
% \maketitle
%
% \tableofcontents
%
% \section{Introduction}
% These packages offer a series of extensions to the standard \LaTeX{}
% \env{tabular} environment.  Their respective functions are:
% \begin{description}
% \item[\Package{multirow}] which provides a construction for table cells
%   that span more than one row of the table;
% \item[\Package{bigstrut}] which creates struts which (slightly) stretch
%   the table row in which they sit.
% \item[\Package{bigdelim}] which creates an appropriately-sized
%   delimiter (for example, brace, parenthesis or bracket) to fit in a
%   single multirow, to indicate a relationship between other rows; and
% \end{description}
%
% \section{Changes in version 2}
% \label{sec:changes-2}
%
%\subsection*{version 2.8}
%
%\begin{itemize}
%\item Optional argument \meta{vmove} for \cs{ldelim} and \cs{rdelim}.
%\end{itemize}
%
%\subsection*{version 2.7}
%
%\begin{itemize}
%\item  Make \cs{@xmultirow} and \cs{multirow@setcolwidth} \cs{long} to allow
% multi-paragraph text.
% See \url{https://github.com/pietvo/multirow/issues/1}
%\end{itemize}
%
%\subsection*{version 2.6}
%
% \begin{itemize}
% \item Make the \texttt{supertabular} option compatible with newer
% versions of the \Package{supertabular} package
% \item Initialize \cs{@arstrutbox} when not defined, to enable some
% uses of the big delims outside of an \env{array} or \env{tabular}.
% \end{itemize}
%
%\subsection*{version 2.5a}
%
% \begin{itemize}
% \item Changed contact information
% \end{itemize}
%
%\subsection*{version 2.5}
%
% \begin{itemize}
% \item Solve a clash with the \Package{colortbl} package, when multirow
% uses the \texttt{longtable} option. There was a clash with both
% packages redefining \cs{@cline}.
% \end{itemize}
%
%\subsection*{version 2.4}
%
% \begin{itemize}
% \item Add a \cs{leavevmode} in \Package{bigstrut} to force horizontal mode
% \item Make \meta{width} and \meta{vmove} in \cs{multirow} \Package{calc} compatible
% \end{itemize}
%
%\subsection*{version 2.3}
%
% \begin{itemize}
% \item Replaced \cs{textrm} with \cs{textnormal} in text beside big braces in bigdelim.sty.
% \end{itemize}
%
%\subsection*{version 2.2}
%
% \begin{itemize}
% \item Support for fractional values of \meta{nrows}.
% \end{itemize}
%
%\subsection*{version 2.0}
%
% \begin{itemize}
% \item \cs{multirow} now has an first optional parameter \oarg{vpos}.
% \item The \meta{width} parameter can be specified as \texttt{=} to use
% the defined width of the column in which the \cs{multirow} appears.
% \item Optional prefix letters (\texttt{t}, \texttt{b}) for
% the \meta{bigstruts} parameter (see section~\ref{sect-refine-bigstrut}).
% \item Package option \texttt{debug}.
% \item Package option \texttt{longtable} to work around a bug in
% \Package{longtable}. See section~\ref{sec:use-with-longtable}.
% \item Package option \texttt{supertabular} to better support \Package{supertabular}. See
% section~\ref{sec:use-with-supertabular}.
% \item Better positioning in some cases.
% \item Lots of documentation.
% \item The distribution is now based on a \texttt{.dtx} file.
% \item Backwards compatible with v1.6.
% \end{itemize}
%
% \section{Using \Package{multirow}}\label{sec:multirow}
%
% \DescribeMacro{\multirow}
% \cs{multirow} sets a piece of text in a \env{tabular} or
% similar environment, spanning multiple rows. We will call the block of
% rows and columns that the text spans the multirow block. Usually this
% covers one column, but by combining it with \cs{multicolumn} more
% columns can be covered.
% \changes{v2.1}{2016/10/06}{Rename \meta{fixup} to \meta{vmove} in the
% documentation as in The LaTeX Companion.}
%
% The basic syntax is:
% \begin{quote}
%   \cmd{\multirow}\oarg{vpos}\marg{nrows}\oarg{bigstruts}\marg{width}\oarg{vmove}\marg{text}
% \end{quote}
% where
% \begin{description}
% \item[\meta{vpos}] defines the vertical positioning of the text in the multirow block.
% The default is \texttt{[c]} which means the text will be vertically centered.
% Other options are \texttt{[t]} for top alignment and \texttt{[b]} for bottom alignment.
% \item[\meta{nrows}] is the number of rows to span.  You should
%   leave the other rows empty at this column, otherwise the stuff created by \cs{multirow}
%   will over-write it. With a positive value of \meta{nrows} the
%   spanned columns are this row and (\meta{nrows}-1) rows below
%   it. With a negative value of \meta{nrows} they are this row and
%   (-\meta{nrows}-1) above it. Fractional values are permitted for
%   \meta{nrows}; this allows for some fine-tuning.
%
% \item[\meta{bigstruts}] is mainly used if you've used the
%   \Package{bigstrut} package.  In that case it is the total number of uses of
%   \cs{bigstrut} within the rows being spanned.  Count 2 uses for each
%   \cs{bigstrut}, 1 for each \cs{bigstrut}\oarg{x} where \meta{x} is
%   either \texttt{t} or \texttt{b}.  The default is 0.
%
% The \meta{bigstruts} parameter can optionally be preceded by a prefix
% letter \texttt{t}, \texttt{b} or \texttt{tb} for finer control. See
% section \ref{sect-refine-bigstrut} for details. The letter may be
% separated from the number by a space character.
%
% \item[\meta{width}] is the width to which the text is to be set.
% Special values are \texttt{*} to indicate that the text parameter's natural width is to
%   be used, and \texttt{=} to indicate that the specified width of the column in
%   which the \cs{multirow} entry is set should be used.
%
% \item[\meta{vmove}] is a length used for fine-tuning: the text will be
%   raised (or lowered, if \meta{vmove} is negative) by that length
%   above (below) wherever it would otherwise have gone.
%
% \item[\meta{text}] is the actual text of the construct.  If the width
%   was set explicitly, the text will be set in a \cs{parbox} of that
%   width; you can use \bsbs{} to force linebreaks where you like.
%
%   If the width was given as \texttt{*} the text will be set in LR
%   mode.  If you want a multiline entry in this case you could use a
%   \env{tabular} or \env{array} environment in the text
%   parameter. See for example the \env{minitab} below.
%
% The width can also be given as \texttt{=} when the \cs{multirow} entry is given in a
% column that has a defined width, for example in a \texttt{p\{\}} column, an
% \texttt{X} column in \env{tabularx} or a \texttt{L}, \texttt{C},
% \texttt{R} or \texttt{J} column in a \env{tabulary} environment.
% The text will be set in a \cs{parbox} of that width.
% If you give ``\texttt{=}'' in other situations, you will get strange
% results (usually a too wide column).
%
% \end{description}
%
% In \Package{multirow} version 2.4 and later, the \meta{width} and
% \meta{vmove} arguments can be given as \Package{calc} expressions if
% the \Package{calc} package is loaded. It is the responsibility of the
% document writer to include the \Package{calc} package;
% \Package{multirow} does not do this.
%
% N.B. \cs{multirow} can be used in the \env{tabular} environment and
% most derivatives of it, for example \env{tabularx},
% \env{tabulary}, \env{supertabular}, \env{ltablex},
% \env{xtab}, \env{longtable}, \env{tabu},
% \env{booktabs} and \env{ctable}. For some of these you
% have to pay special attention to certain cases, see below.
%
% \DescribeMacro{\multirowsetup}
% Just before \meta{text} is expanded, the \cs{multirowsetup} macro is
% expanded to set up any special environment.  Initially,
% \cs{multirowsetup} contains just \cs{raggedright}.  It may be
% redefined with \cs{renewcommand}.
%
% If you want to use both \cs{multirow} and \cs{multicolumn}
% \index{multicolumn=\verb!*+\multicolumn+|usage}
% on the same entry, you must put the \cs{multirow} inside the
% \cs{multicolumn}. The other way around will not work.
% For example:
% \begin{verbatim}
%    \multicolumn{2}{c}{\multirow{3}{*}{Multi-multi}}
% \end{verbatim}
%
% \subsection{Package Options}
% \label{sec:package-options}
%
% \DescribeMacro{multirowdebugtrue}
% \DescribeMacro{multirowdebugfalse}
% The following options are defined:
% \begin{description}
% \item[debug]
% This option causes information about multirow boxes
% to be written to the log file. This is done by the \TeX{} \cs{showbox}
% command. Note: this will cause the \LaTeX{} compilation to be considered
% failed, even if there is no real error.
% This option can also be activated anywhere in the document with the
% command \cs{multirowdebugtrue} and deactivated with
% \cs{multirowdebugfalse}. When such a command is placed just before a
% \cs{multirow}, it applies only to that specific \cs{multirow} entry.
% \item[longtable]
% The \texttt{longtable} option redefines the \cs{cline} macro to work
% around a bug in the \Package{longtable} package. See section \ref{sec:use-with-longtable}.
% \end{description}
%
%\subsection{Examples}
%\label{sec:examples}
%
% An example with both \Package{multirow} and \Package{bigstrut}):
%
% \begin{quote}
% \begin{verbatim}
% \newcommand{\minitab}[2][l]{\begin{tabular}{#1}#2\end{tabular}}
% \begin{tabular}{|c|c|}
% \hline
% \multirow{4}{1in}{Common g text} & Column g2a\\
%       & Column g2b \\
%       & Column g2c \\
%       & Column g2d \\
% \hline
% \multirow{3}[6]*{Common g text} & Column g2a\bigstrut\\\cline{2-2}
%       & Column g2b \bigstrut\\\cline{2-2}
%       & Column g2c \bigstrut\\
% \hline
% \multirow{4}[8]{1in}{Common g text, but a bit longer.} &
%                                    Column g2a\bigstrut\\\cline{2-2}
%       & Column g2b \bigstrut\\\cline{2-2}
%       & Column g2c \bigstrut\\\cline{2-2}
%       & Column g2d \bigstrut\\
% \hline
% \multirow{4}*{\minitab[c]{Common \\ g text}} & Column g2a\\
%       & Column g2b \\
%       & Column g2c \\
%       & Column g2d \\
% \hline
% \end{tabular}
% \end{verbatim}
% \end{quote}
% which will appear as:
% \begin{quote}
%   \begin{tabular}{|c|c|}
%     \hline
%     \multirow{4}{1in}{Common g text} & Column g2a\\
%       & Column g2b \\
%       & Column g2c \\
%       & Column g2d \\
%     \hline
%     \multirow{3}[6]*{Common g text} & Column g2a\bigstrut\\\cline{2-2}
%       & Column g2b \bigstrut\\\cline{2-2}
%       & Column g2c \bigstrut\\
%     \hline
%     \multirow{4}[8]{1in}{Common g text, but a bit longer.} &
%                                        Column g2a\bigstrut\\\cline{2-2}
%       & Column g2b \bigstrut\\\cline{2-2}
%       & Column g2c \bigstrut\\\cline{2-2}
%       & Column g2d \bigstrut\\
%     \hline
%     \multirow{4}*{\minitab[c]{Common \\ g text}} & Column g2a\\
%       & Column g2b \\
%       & Column g2c \\
%       & Column g2d \\
%     \hline
%   \end{tabular}
%   \quad
%   \begin{tabular}{ll}
%     \multirow{4}{*}{Normal case} & \\
%       & \\
%       & \\
%       & \\
%     \multirow{3}[6]*{With \cs{bigstrut}s and \texttt{*} as \meta{width}} & \bigstrut\\
%       & \bigstrut\\
%       & \bigstrut\\
%     \multirow{4}[8]{*}{With \cs{bigstrut}s and normal \meta{width}} & \bigstrut\\
%       & \bigstrut\\
%       & \bigstrut\\
%       & \bigstrut\\
%     \multirow{4}*{Multiline text in \cs{multirow}} & \\
%       & \\
%       & \\
%       & \\
%   \end{tabular}
%
% \end{quote}
% \vspace{1ex}
% An example with the ``\texttt{=}'' \meta{width} specifier in a
% \env{tabulary} (Note: The braces around the \texttt{=} may be omitted):
%
% \begin{quote}
% \index{extrarowheight=\verb!*+\extrarowheight+|usage}
% \begin{verbatim}
% \setlength{\extrarowheight}{2pt}
% \begin{tabulary}{11cm}{|L|L|L|}
%   \hline
%   All human beings are born free and equal in dignity and rights. &
%      \multirow{2}{=}{Everyone is entitled to all the rights and
%      freedoms set forth in this Declaration, without distinction of
%      any kind, such as race, colour, sex, language, religion,
%      political or other opinion, national or social origin, property,
%      birth or other status.}
%   & Everyone has the right to life, liberty and security of person. \\
% \cline{1-1}\cline{3-3}
%      No one shall be held in slavery or servitude; slavery and the
%      slave trade shall be prohibited in all their forms. &
%   & No one shall be subjected to torture or to cruel, inhuman or
%   degrading treatment or punishment. \\
% \hline
% \end{tabulary}
% \end{verbatim}
% \end{quote}
%
% \begin{quote}
% \index{extrarowheight=\verb!*+\extrarowheight+|usage}
% \setlength{\extrarowheight}{2pt}
% \begin{tabulary}{11cm}{|L|L|L|}
%   \hline
%   All human beings are born free and equal in dignity and rights. &
%      \multirow{2}{=}{Everyone is entitled to all the rights and
%      freedoms set forth in this Declaration, without distinction of
%      any kind, such as race, colour, sex, language, religion,
%      political or other opinion, national or social origin, property,
%      birth or other status.}
%   & Everyone has the right to life, liberty and security of person. \\
% \cline{1-1}\cline{3-3}
%      No one shall be held in slavery or servitude; slavery and the
%      slave trade shall be prohibited in all their forms. &
%   & No one shall be subjected to torture or to cruel, inhuman or
%   degrading treatment or punishment. \\
% \hline
% \end{tabulary}
% \end{quote}
%
% A few observations about this example:
% \begin{enumerate}
% \item  The middle column is the \cs{multirow}. You would expect it to
% be vertically centered, but it isn't. This is because \cs{multirow}
% doesn't know the height of the box. The only estimate \cs{multirow}
% can make about the height is the number of rows${}\times{}$the normal
% height of a row. It tries to center the text in that space, but that space
% is too low in this example. Therefore the text is at the top of the
% box. If you want it to be centered, you have to supply a \meta{vmove}
% argument to shift it down.
% \item We have used an \cs{extrarowheight}\footnote{This is only
% available with the \Package{array} package, which \Package{tabulary}
% includes automatically.} of 2pt, to make a bit room between the
% \cs{hline} and the following text. However, this is not applied to the
% \cs{multirow}, because this is thought to be centered. In this case
% you can give the \meta{vpos} argument as \texttt{[t]}, in which case
% \cs{multirow} will do the proper positioning.
% \end{enumerate}
%
% Now with a negative \meta{nrows}.
%
% \begin{quote}
% \index{extrarowheight=\verb!*+\extrarowheight+|usage}
% \begin{verbatim}
% \setlength{\extrarowheight}{2pt}
% \begin{tabulary}{11cm}{|L|L|L|}
%   \hline
%   All human beings are born free and equal in dignity and rights. &
%   & Everyone has the right to life, liberty and security of person. \\
% \cline{1-1}\cline{3-3}
%      No one shall be held in slavery or servitude; slavery and the
%      slave trade shall be prohibited in all their forms. &
%      \multirow{-2}{=}[12mm]{Everyone is entitled to all the rights and
%      freedoms set forth in this Declaration, without distinction of
%      any kind, such as race, colour, sex, language, religion,
%      political or other opinion, national or social origin, property,
%      birth or other status.}
%   & No one shall be subjected to torture or to cruel, inhuman or
%   degrading treatment or punishment. \\
% \hline
% \end{tabulary}
% \end{verbatim}
% \end{quote}
%
% \begin{quote}
% \index{extrarowheight=\verb!*+\extrarowheight+|usage}
% \setlength{\extrarowheight}{2pt}
% \begin{tabulary}{11cm}{|L|L|L|}
%   \hline
%   All human beings are born free and equal in dignity and rights. &
%   & Everyone has the right to life, liberty and security of person. \\
% \cline{1-1}\cline{3-3}
%      No one shall be held in slavery or servitude; slavery and the
%      slave trade shall be prohibited in all their forms. &
%      \multirow{-2}{=}[12mm]{Everyone is entitled to all the rights and
%      freedoms set forth in this Declaration, without distinction of
%      any kind, such as race, colour, sex, language, religion,
%      political or other opinion, national or social origin, property,
%      birth or other status.}
%   & No one shall be subjected to torture or to cruel, inhuman or
%   degrading treatment or punishment. \\
% \hline
% \end{tabulary}
% \end{quote}
%
% In this case the text would be centered somewhere in the
% bottom row, which would make it stick out of the bottom. Therefore we
% applied a \meta{vmove} of 12mm. The \meta{vmove} usually requires some experimentation.
%
% \subsection{Fine-Tuning}
% \label{sec:fine-tuning}
%
% If any of the spanned rows are unusually large, or if you're using the
% \Package{bigstrut} package and \cs{bigstrut}s are used asymetrically about the
% centerline of the spanned rows, the vertical centering may not come
% out right.  Use the \meta{vmove} parameter in this case. Sometimes it
% may be more helpful to just use a larger value for \meta{nrows},
% including fractional values. See an example in section~\ref{sec:dealing-with-tall}.
%
% It's just about impossible to deal correctly with descenders.  The
% text will be set up centered, but it may then have a baseline that
% doesn't match the baseline of the stuff beside it, in particular if
% the stuff beside it has descenders and \meta{text} does not.  This may
% result in a small misalignment.  About all that can be done is to do a
% final touchup on \meta{text}, using the \meta{vmove} optional parameter.
% (Hint:  If you use a measure like \texttt{.1ex}, there's a reasonable
% chance that the \meta{vmove} will still be correct if you change the point
% size.)
%
% \cs{multirow} is mainly designed for use with \env{tabular}, as
% opposed to \env{array}, environments.  It might not work well in an
% array environment if there are big formulas in some rows; in that case you can use the
% \meta{vmove} parameter to refine the result.
%
% In some cases you might want to align the multirow entry with the top
% of the other row cells, for example if you have a large capital in it.
% when you use \meta{vpos} = \texttt{[t]}, the baselines will be
% aligned, which is the wrong thing in this case. You can then do the
% positioning with the \meta{vmove} parameter and let \LaTeX{} calculate
% the amount. For example:
%
% \begin{quote}
% \begin{verbatim}
% \usepackage{calc}
% \newlength{\shiftdown}
% \setlength{\shiftdown}{\heightof{\Huge\bfseries B}-\heightof{f}}
% . . .
% \begin{tabular}{cll}
%   \toprule
%   \multirow[t]{5}{*}[-\shiftdown]{\Huge\bfseries B}
%                     & foo  & Lorem ipsum dolor sit    \\
%                     & bar  & Maecenas sed purus       \\
%                     & baz  & Nullam luctus id tellus  \\
%                     & qux  & Aenean consequat commodo \\
%   \bottomrule
% \end{tabular}
% \end{verbatim}
% \end{quote}
%
% \begin{quote}
% \newlength{\shiftdown}
% \setlength{\shiftdown}{\heightof{\Huge\bfseries B}-\heightof{f}}
% \begin{tabular}{cll}
%   \toprule
%   \multirow[t]{5}{*}[-\shiftdown]{\Huge\bfseries B}
%                     & foo  & Lorem ipsum dolor sit    \\
%                     & bar  & Maecenas sed purus       \\
%                     & baz  & Nullam luctus id tellus  \\
%                     & qux  & Aenean consequat commodo \\
%   \bottomrule
% \end{tabular}
% \end{quote}
% In \Package{multirow} version 2.4 and later you can also directly use the expression
% \verb|-\heightof{\Huge\bfseries B}-\heightof{f}| instead of \verb|-\shiftdown| for the \meta{vmove} argument.
%
% \subsection{Multirow and colored cells}\label{sec:mult-color-cells}
%
% If you use \cs{multirow} with the \Package{colortbl} package (or the
% \index{colortbl=\Package{colortbl}|usage}
% \Package{xcolor} package with the \texttt{[table]} option) you have
% \index{xcolor=\Package{xcolor}|usage}
% to take precautions if you want to color the column that has the
% \cs{multirow} in it.  The \Package{colortbl} package works by coloring each cell
% separately.  So if you use \cs{multirow} with a positive \meta{nrows}
% value, \Package{colortbl} will first color the top cell, then
% \cs{multirow} will typeset \meta{nrows} cells starting with this cell,
% and later \Package{colortbl} will color the other cells, effectively
% hiding the text in that area.  This can be solved by putting the
% \cs{multirow} in the last row with a negative \meta{nrows} value.
% See, for example:
% \begin{quote}
% \begin{verbatim}
% \begin{tabular}{l>{\columncolor{yellow}}l}
%   aaaa & \\
%   cccc & \\
%   dddd & \multirow{-3}*{bbbb}\\
% \end{tabular}
% \end{verbatim}
% \end{quote}
% which will produce:
% \begin{quote}
%   \begin{tabular}{l>{\columncolor{yellow}}l}
%     aaaa & \\
%     cccc & \\
%     dddd & \multirow{-3}*{bbbb}\\
%   \end{tabular}
% \end{quote}
%
% \index{hhline=\Package{hhline}|usage}
% When you use colored multirow cells together with the
% \Package{hhline} package you may find some white stripes in your
% colored multirow cell. For example:
% \begin{quote}
% \begin{verbatim}
%   \begin{tabular}{|>{\columncolor{red}}c|c|}
%   \hline
%   \bfseries ColumnOne & \bfseries ColumnTwo\\ \hline
%   First data & 932\\ \hline
%    & 239\\ \hhline{|~|-|}
%    & 137\\ \hhline{|~|-|}
%   \multirow{-3}{*}{More data} & 319\\ \hline
%   Last data & 132\\ \hline
% \end{tabular}
% \end{verbatim}
% \end{quote}
%
% \begin{quote}
%   \begin{tabular}{|>{\columncolor{red}}c|c|}
%   \hline
%   \bfseries ColumnOne & \bfseries ColumnTwo\\ \hline
%   First data & 932\\ \hline
%    & 239\\ \hhline{|~|-|}
%    & 137\\ \hhline{|~|-|}
%   \multirow{-3}{*}{More data} & 319\\ \hline
%   Last data & 132\\ \hline
% \end{tabular}
% \end{quote}
%
% This can be solved by putting colored horizontal rules with the same
% color in the colored multirow cell.
% \begin{quote}
% \begin{verbatim}
%   \begin{tabular}{|>{\columncolor{red}}c|c|}
%   \hline
%   \bfseries ColumnOne & \bfseries ColumnTwo\\ \hline
%   First data & 932\\ \hline
%    & 239\\ \hhline{|>{\arrayrulecolor{red}}->{\arrayrulecolor{black}}|-|}
%    & 137\\ \hhline{|>{\arrayrulecolor{red}}->{\arrayrulecolor{black}}|-|}
%   \multirow{-3}{*}{More data} & 319\\ \hline
%   Last data & 132\\ \hline
% \end{tabular}
% \end{verbatim}
% \end{quote}
%
% \begin{quote}
%   \begin{tabular}{|>{\columncolor{red}}c|c|}
%   \hline
%   \bfseries ColumnOne & \bfseries ColumnTwo\\ \hline
%   First data & 932\\ \hline
%    & 239\\ \hhline{|>{\arrayrulecolor{red}}->{\arrayrulecolor{black}}|-|}
%    & 137\\ \hhline{|>{\arrayrulecolor{red}}->{\arrayrulecolor{black}}|-|}
%   \multirow{-3}{*}{More data} & 319\\ \hline
%   Last data & 132\\ \hline
% \end{tabular}
% \end{quote}
%
% \subsection{Fine-tuning the \meta{bigstruts} argument}
% \label{sect-refine-bigstrut}
%
% \cs{multirow} can calculate the height of the required multirow box
% from \meta{nrows} and \meta{bigstruts}, supposed that all the rows
% don't have  ``unusual'' heights. However, there are cases when this is
% not enough to properly position the box, especially when there is a
% \cs{bigstrut} on top of the first row and/or one on the bottom of the
% last row. In that case \cs{multirow} should be given additional
% information. This is done by prefixing the \meta{bigstruts} argument
% with a letter (or two) indicating which of these two are present.
%
% See the following examples:\\ (in these examples we have
% \verb+\setlength{\bigstrutjot}{10pt}+ to make the effect clearly visible)\\[5mm]
% \begin{minipage}[c]{0.23\linewidth}
%   \setlength{\bigstrutjot}{10pt}
%   \begin{tabular}{|c|c|}
%     \hline
%     \multirow{3}[1]{*}{Multirow} & T \bigstrut[t] \\
%     \cline{2-2}
%     &  X \\
%     \cline{2-2}
%     &  B \\
%     \hline
%   \end{tabular} \\[2mm]
%   \begin{tabular}{|c|c|}
%     \hline
%     \multirow{3}[t 1]{*}{Multirow} & T \bigstrut[t] \\
%     \cline{2-2}
%     &  X \\
%     \cline{2-2}
%     &  B \\
%     \hline
%   \end{tabular}
% \end{minipage}
% \begin{minipage}[c]{0.75\linewidth}
% \begin{verbatim}
%   \begin{tabular}{|c|c|}
%     \hline
%     \multirow{3}[1]{*}{Multirow} & T \bigstrut[t] \\
%     \cline{2-2}
%     &  X \\
%     \cline{2-2}
%     &  B \\
%     \hline
%   \end{tabular}
% \end{verbatim}
% \end{minipage}
% \\[5mm]
% In the top box in the above example the text ``Multirow'' should be centered, but it
% is a bit below the center, because of the \cs{bigstrut[t]} in the top
% row. We can correct this by giving the \meta{bigstruts} parameter as
% ``t 1'', indicating a bigstrut in the top. This is done in the bottom box, where
% \verb+\multirow{3}[t 1]{*}{Multirow}+ is used.
%
% A second example:\\[5mm]
% \begin{minipage}{0.23\linewidth}
%   \setlength{\bigstrutjot}{10pt}
%   \begin{tabular}{|c|c|}
%     \hline
%     & T \\
%     \cline{2-2}
%     &  X \\
%     \cline{2-2}
%     \multirow[t]{-3}[1]{*}{Multirow} &  B \bigstrut[b] \\
%     \hline
%   \end{tabular} \\[2mm]
%   \begin{tabular}{|c|c|}
%     \hline
%     & T \\
%     \cline{2-2}
%     &  X \\
%     \cline{2-2}
%     \multirow[t]{-3}[b 1]{*}{Multirow} &  B \bigstrut[b] \\
%     \hline
%   \end{tabular}
% \end{minipage}
% \begin{minipage}{0.75\linewidth}
% \begin{verbatim}
%   \begin{tabular}{|c|c|}
%     \hline
%     & T \\
%     \cline{2-2}
%     &  X \\
%     \cline{2-2}
%     \multirow[t]{-3}[1]{*}{Multirow} &  B \bigstrut[b] \\
%     \hline
%   \end{tabular}
% \end{verbatim}
% \end{minipage}
% \\[5mm]
% In the top box the \verb+\multirow[t]+ should be positioned on the
% same height as the \texttt{T}, but it is too high, because there is a
% \cs{bigstrut} in the bottom. We can correct that by specifying the
% \meta{bigstruts} argument as ``b 1'', i.e. using
% \verb+\multirow[t]{-3}[b 1]{*}{Multirow}+.
% \\[5mm]
% The possibilities for the prefix are:
% \begin{description}
% \item[\texttt{t}] There is a bigstrut in the top, i.e. a \cs{bigstrut}
% or \cs{bigstrut[t]} in the top row.
% \item[\texttt{b}] There is a bigstrut in the bottom, i.e. a \cs{bigstrut}
% or \cs{bigstrut[b]} in the bottom row.
% \item[\texttt{tb}] They are both present. Note: this cannot be given
% as \texttt{bt}.
% \end{description}
% The space between the letter(s) and the number is optional.
% Please note that the prefix does not depend on whether the
% \cs{multirow} is in the top or the bottom row.
%
% \subsection{Use with \Package{longtable}}
% \label{sec:use-with-longtable}
%
% \index{longtable=\env{longtable}|usage}
% \index{longtabu=\env{longtabu}|usage}
% It is possible to use \cs{multirow} in a \env{longtable}
% environment (as well as in its descendent \env{longtabu}).
% However, care must be taken that the longtable doesn't break the
% multirow entry when it is near the bottom of the page. For example:
%
% \noindent
% \begin{minipage}[t]{0.4\linewidth}
% \begin{longtable}{|l|l|l|}
% \ldots & \ldots & \ldots \\
% \hline
% Sept. 21 & 09:00 & event 1 \\
% \hline
% Sept. 22 & 10:00 & event 2 \\
% \hline
% \multirow{3}*{Sept. 23} & 10:00 & event 3 \\
%                         & 12:00 & event 4 \\
%                         & 15:00 & event 5 \\
% \hline
% Sept. 24 & 09:00 & event 6 \\
% \hline
% \ldots & \ldots & \ldots \\
% \end{longtable}
% \end{minipage}
% \begin{minipage}[t]{0.58\linewidth}
% \begin{verbatim}
% \begin{longtable}{|l|l|l|}
% \ldots & \ldots & \ldots \\
% \hline
% Sept. 21 & 09:00 & event 1 \\
% \hline
% Sept. 22 & 10:00 & event 2 \\
% \hline
% \multirow{3}*{Sept. 23} & 10:00 & event 3 \\
%                         & 12:00 & event 4 \\
%                         & 15:00 & event 5 \\
% \hline
% Sept. 24 & 09:00 & event 6 \\
% \hline
% \ldots & \ldots & \ldots \\
% \end{longtable}
% \end{verbatim}
% \end{minipage}
%
% In this case if the ``Sept. 23'' entry comes close to the bottom of the
% page, you want to prevent the pagebreak to occur in the middle of this
% entry. You can do this by ending the intermediate rows with \verb+\\*+ instead of \verb+\\+.
%
% \begin{verbatim}
% \multirow{3}*{Sept. 23} & 10:00 & event 3 \\*
%                         & 12:00 & event 4 \\*
%                         & 15:00 & event 5 \\
% \end{verbatim}
%
% There is, however, a bug in \Package{longtable}, that causes the
% \verb+\\*+ not to work if it is followed by a \cs{cline}, like in the
% following example:
%
% \begin{verbatim}
% \multirow{3}*{Sept. 23} & 10:00 & event 3 \\*
% \cline{2-3}
%                         & 12:00 & event 4 \\*
% \cline{2-3}
%                         & 15:00 & event 5 \\
% \end{verbatim}
%
% \Package{multirow} has a package option \texttt{longtable} that redefines
% \cs{cline} so that the \verb+\\*+ will also work when followed by
% \cs{cline}. The code comes from David Carlisle.
%
% \subsection{Use with \Package{supertabular}}
% \label{sec:use-with-supertabular}
%
% \index{supertabular=\env{supertabular}|usage}
% \index{xtab=\env{xtab}|usage}
% With the package \Package{supertabular} (or the augmented version
% \Package{xtab}) there is the same requirement
% to keep the rows of a multirow together when a pagebreak occurs.
% Unfortunately, \Package{supertabular} does not have a way to specify
% that a pagebreak should be suppressed. I.e. \verb+\\*+ does not
% suppress a pagebreak.
% Therefore \Package{multirow} provides a package option
% \texttt{supertabular} that redefines \verb+\\*+ inside a
% \env{supertabular} to suppress the pagebreak. You should use
% this to end the intermediate rows in a multirow block. However, this
% does not cause \env{supertabular} to consider breaking the
% page before the \cs{multirow}, contrary to \env{longtable}.
% Thus the table may become too long.
%
% \DescribeMacro{\STneed}
% Therefore when the \texttt{supertabular} option is given,
% \Package{multirow} also provides a command \cs{STneed} to be used
% in a \env{supertabular} that specifies how much space we need
% on the page. Then if there is not enough space, a pagebreak will occur
% at that place. For example:
% \begin{verbatim}
% \tabletail{\hline}
% \begin{supertabular}{|l|l|l|}
% \ldots & \ldots & \ldots \\
% \hline
% Sept. 20 & 10:00 & event 1\\
% \hline
% Sept. 21 & 10:00 & event 2\\
% \hline
% Sept. 22 & 10:00 & event 3\\
% \hline
% \STneed {2cm}
% \multirow{3}*{Sept. 23} & 10:00 & event 4\\*
% \cline{2-3}
%                         & 12:00 & event 5 \\*
% \cline{2-3}
%                         & 15:00 & event 6 \\
% \hline
%  Sept. 24 & 09:00 & event 7 \\
% \hline
% \ldots & \ldots & \ldots \\
% \end{verbatim}
%
% You will have to experiment a bit with the value to see
% what works. Sometimes it is better to exaggerate the required space a bit.
%
% \subsection{Dealing with tall entries}
% \label{sec:dealing-with-tall}
%
% Sometimes there are rows that are taller than what is expected. This
% section gives some hints how to deal with these situations. There are
% two cases:
% \begin{enumerate}
% \item When there is an exceptionally tall row outside of the multirow
% block the positioning of the \cs{multirow} might be wrong. This is
% because \cs{multirow} does not have information about the heights
% of the rows. This can happen for example when a large formula is
% entered in a cell, or a multi-line paragraph in a \texttt{[\{\}]}
% column. An example:
%
% \begin{quote}
% \begin{verbatim}
% \begin{tabular}{| l | l | p{4cm} |}
%   \hline
%   \multirow{3}*{Week 38} & Monday & Rain most of the day\\
%   \cline{2-3}
%                     & Tuesday & Sunny with some clouds\\
%   \cline{2-3}
%                     & Wednesday & A clear day with a lot of sunshine.
%                     However, the strong wind will bring down the
%                     temperature. \\
%   \hline
% \end{tabular}
% \end{verbatim}
%  \end{quote}
% \begin{quote}
% \begin{tabular}{| l | l | p{4cm} |}
%   \hline
%   \multirow{3}*{Week 38} & Monday & Rain most of the day\\
%   \cline{2-3}
%                     & Tuesday & Sunny with some clouds\\
%   \cline{2-3}
%                     & Wednesday & A clear day with a lot of sunshine.
%                     However, the strong wind will bring down the
%                     temperature. \\
%   \hline
% \end{tabular}
% \end{quote}
% The \cs{multirow} is positioned on the second row, because it
% specifies that it should cover 3 rows. However, the second row is not
% the vertical center in this case because the third row is much taller.
%
% To remedy this, the \meta{vmove} parameter could be used. However, in
% this case it would be easier to pretend that \cs{multirow} spans 6
% rows (the total number of lines in the last column). So use
% \verb+\multirow{6}+\ldots{} and we get:
% \begin{quote}
% \begin{tabular}{| l | l | p{4cm} |}
%   \hline
%   \multirow{6}*{Week 38} & Monday & Rain most of the day\\
%   \cline{2-3}
%                     & Tuesday & Sunny with some clouds\\
%   \cline{2-3}
%                     & Wednesday & A clear day with a lot of sunshine.
%                     However, the strong wind will bring down the
%                     temperature. \\
%   \hline
% \end{tabular}
% \end{quote}
% \item The second case is when the \cs{multirow} entry is taller than
% the surrounding normal rows. In that case the multirow text will stick
% out of its block. We must now enlarge the other rows, and that is
% something \cs{multirow} cannot do.
%
% An example:
% (Don't take this as a medical advice. The names are fake anyway.)
% \begin{verbatim}
% \begin{tabular}{| p{2mm} l | p{5cm} |}
%  \hline
%  \multicolumn{2}{|l|}{\textbf{Medicine \& dose}}
%                               & \textbf{Possible Side effects} \\
%  \hline
%  \multicolumn{2}{|l|}{Spirino}
%                      & \multirow{3}={Confusion,
%                        hallucinations, rapid breathing,
%                        seizure (convulsions);
%                        upset stomach, heartburn; severe nausea,
%                        vomiting, or stomach pain or mild headache.} \\
%  \cline{1-2}
%   & initial: 200 mg/day & \\
%  \cline{1-2}
%    & maintenance: 100-400 mg/day & \\
%  \hline
%  \multicolumn{2}{|l|}{Conzac}
%                      & \multirow{3}={Anxiety; nervousness;
%                       insomnia; anorexia; mild bradycardia;
%                       SA node slowing; weight loss;
%                       solar photosensitivity; hyponatremia;
%                       sexual dysfunction (both genders); may
%                       alter glycemic control in diabetic patients.} \\
%  \cline{1-2}
%   & initial: 10 mg/day & \\
%  \cline{1-2}
%    & maintenance: 10-40 mg/day & \\
%  \hline
% \end{tabular}
% \end{verbatim}
% \begin{tabular}{| p{2mm} l | p{5cm} |}
%  \hline
%  \multicolumn{2}{|l|}{\textbf{Medicine \& dose}}
%                               & \textbf{Possible Side effects} \\
%  \hline
%  \multicolumn{2}{|l|}{Spirino}
%                      & \multirow{3}={Confusion,
%                        hallucinations, rapid breathing,
%                        seizure (convulsions);
%                        upset stomach, heartburn; severe nausea,
%                        vomiting, or stomach pain or mild headache.} \\
%  \cline{1-2}
%   & initial: 200 mg/day & \\
%  \cline{1-2}
%    & maintenance: 100-400 mg/day & \\
%  \hline
%  \multicolumn{2}{|l|}{Conzac}
%                      & \multirow{3}={Anxiety; nervousness;
%                       insomnia; anorexia; mild bradycardia;
%                       SA node slowing; weight loss;
%                       solar photosensitivity; hyponatremia;
%                       sexual dysfunction (both genders); may
%                       alter glycemic control in diabetic patients.} \\
%  \cline{1-2}
%   & initial: 10 mg/day & \\
%  \cline{1-2}
%    & maintenance: 10-40 mg/day & \\
%  \hline
% \end{tabular}
% \\[2cm]
% Both \cs{multirow} entries are too high; the first sticks out into the
% second entry, and the second one sticks out of the table.
%
% There are two ways we can correct this:
% The simplest would be to add extra empty rows to cover the overlapping
% space. For the first entry that would be 2 extra rows; for the second
% 4. So we add twice \verb+ & & \\+ before the third \verb+\hline+,
% and four of these before the last \verb+\hline+.
% This gives us just the correct table:
% \\[5mm]
% \begin{tabular}{| p{2mm} l | p{5cm} |}
%   \hline
%   \multicolumn{2}{|l|}{\textbf{Medicine \& dose}}
%                               & \textbf{Possible Side effects} \\
%   \hline
%  \multicolumn{2}{|l|}{Spirino}
%                      & \multirow{3}={Confusion,
%                        hallucinations, rapid breathing,
%                        seizure (convulsions);
%                        upset stomach, heartburn; severe nausea,
%                        vomiting, or stomach pain or mild headache.} \\
%   \cline{1-2}
%     & initial: 200 mg/day & \\
%   \cline{1-2}
%      & maintenance: 100-400 mg/day & \\
%     & & \\
%     & & \\
%   \hline
%  \multicolumn{2}{|l|}{Conzac}
%                      & \multirow{3}={Anxiety; nervousness;
%                       insomnia; anorexia; mild bradycardia;
%                       SA node slowing; weight loss;
%                       solar photosensitivity; hyponatremia;
%                       sexual dysfunction (both genders); may
%                       alter glycemic control in diabetic patients.} \\
%   \cline{1-2}
%     & initial: 10 mg/day & \\
%   \cline{1-2}
%      & maintenance: 10-40 mg/day & \\
%     & & \\
%     & & \\
%     & & \\
%     & & \\
%   \hline
% \end{tabular}
%
% The second way is to stretch the normal rows vertically, such that
% they fit with the multirow entry. In this table, where the font size
% is 10pt,  each row has a total
% height of 12pt. For the first entry we need 24pt extra (2 rows),
% Because this space must be divided over 3 rows that is 8pt per row,
% making the total height of the row 20pt.
% The normal row has a height of 8.4pt and a depth of 3.6pt (total
% 12pt). We can add 4pt on the top and 4pt on the bottom, or any other
% combination that adds up to 8pt. In this case I have chosen to make
% the height 12pt and the depth 8pt. We do this with a \cs{rule} with 0 width.
% \DescribeMacro{\mystrut}
% \verb+\newcommand{\mystrut}{\rule[-8pt]{0pt}{20pt}}+
% and put \cs{mystrut} in each of the first 3 rows. By defining your own
% struts you have complete control over the layout. You can choose to
% give some rows more space than others, or to put all the space is the
% last row, for example.
%
% For the second entry we need 48pt extra (4 rows). We will use
% \cs{bigstrut} in each row, that is 16pt per row, and as a \cs{bigstrut} is
% 2\cs{bigstrutjot}s, we set \cs{bigstrutjot} to 8pt. The
% \Package{booktabs} package adds some extra vertical space around the rules,
% therefore when using the normal \env{tabular} environment, it
% is probably better to make the struts a little bit bigger, or a bit
% smaller with \Package{booktabs}. After some experimentation it appeared
% that a \cs{bigstrutjot} of \texttt{7pt} was enough. Of course we added
% the \meta{bigstruts} argument of \texttt{[tb6]} to the second
% multirow. Please note that this is not possible with our own struts,
% unless we cheat.
%
% \index{booktabs=\env{booktabs}|usage}
% Now with \Package{booktabs} the code becomes:
% \begin{verbatim}
% \newcommand{\mystrut}{\rule[-8pt]{0pt}{20pt}}
% \setlength{\bigstrutjot}{7pt}
% \begin{tabular}{ p{2mm} l  p{5cm} }
%  \toprule
%  \multicolumn{2}{l}{\textbf{Medicine \& dose}}
%                               & \textbf{Possible Side effects} \\
%  \cmidrule(r){1-2} \cmidrule(l){3-3}
%  \multicolumn{2}{l}{Spirino} \mystrut
%                      & \multirow{3}={Confusion,
%                        hallucinations, rapid breathing,
%                        seizure (convulsions);
%                        upset stomach, heartburn; severe nausea,
%                        vomiting, or stomach pain or mild headache.} \\
%  \cmidrule(r){1-2}
%     & initial: 200 mg/day \mystrut & \\
%  \cmidrule(r){1-2}
%      & maintenance: 100-400 mg/day \mystrut & \\
%  \midrule
%  \multicolumn{2}{l}{Conzac} \bigstrut
%                      & \multirow{3}[tb6]={Anxiety; nervousness;
%                       insomnia; anorexia; mild bradycardia;
%                       SA node slowing; weight loss;
%                       solar photosensitivity; hyponatremia;
%                       sexual dysfunction (both genders); may
%                       alter glycemic control in diabetic patients.} \\
%  \cmidrule(r){1-2}
%    & initial: 10 mg/day \bigstrut & \\
%  \cmidrule(r){1-2}
%    & maintenance: 10-40 mg/day \bigstrut  & \\
%  \bottomrule
% \end{tabular}
% \end{verbatim}
%
% \newcommand{\mystrut}{\rule[-8pt]{0pt}{20pt}}
% \setlength{\bigstrutjot}{7pt}
% \begin{tabular}{ p{2mm} l  p{5cm} }
%  \toprule
%  \multicolumn{2}{l}{\textbf{Medicine \& dose}}
%                               & \textbf{Possible Side effects} \\
%  \cmidrule(r){1-2} \cmidrule(l){3-3}
%  \multicolumn{2}{l}{Spirino} \mystrut
%                      & \multirow{3}={Confusion,
%                        hallucinations, rapid breathing,
%                        seizure (convulsions);
%                        upset stomach, heartburn; severe nausea,
%                        vomiting, or stomach pain or mild headache.} \\
%  \cmidrule(r){1-2}
%     & initial: 200 mg/day \mystrut & \\
%  \cmidrule(r){1-2}
%      & maintenance: 100-400 mg/day \mystrut & \\
%  \midrule
%  \multicolumn{2}{l}{Conzac} \bigstrut
%                      & \multirow{3}[tb6]={Anxiety; nervousness;
%                       insomnia; anorexia; mild bradycardia;
%                       SA node slowing; weight loss;
%                       solar photosensitivity; hyponatremia;
%                       sexual dysfunction (both genders); may
%                       alter glycemic control in diabetic patients.} \\
%  \cmidrule(r){1-2}
%    & initial: 10 mg/day \bigstrut & \\
%  \cmidrule(r){1-2}
%    & maintenance: 10-40 mg/day \bigstrut  & \\
%  \bottomrule
% \end{tabular}
% \end{enumerate}
%
% \section{Using \Package{bigstrut}}
%
% \DescribeMacro{\bigstrut}
% \DescribeMacro{\bigstrutjot}
% \cs{bigstrut}\oarg{x} produces a strut (a rule with width 0) which is \cs{bigstrutjot}
% (\texttt{2pt} by default) higher, lower, or both than the standard
% \env{array}/\env{tabular} strut.  Use it in table entries that are adjacent to
% \cs{hline}s to leave an extra bit of space\,---\,according to the
% TeXbook (page 246), ``This is a little touch that improves the
% appearance of boxed tables; look for it as a mark of quality.''
%
% Although you could use \cs{bigstrut} in an array, there isn't normally
% much point since arrays are `opened up' by \cs{jot} anyway.
%
% \cs{bigstrut[t]} adds height; \cs{bigstrut[b]}  adds
% depth.  Just \cs{bigstrut} adds both.  So:  Use
% \cs{bigstrut[t]} in the row just \emph{after} an \cs{hline};
% \cs{bigstrut[b]} in the row just \emph{before}; and
% \cs{bigstrut} if there are \cs{hline}s both before and after.
%
% Spaces after the \cs{bigstrut} are ignored, even if it has an optional
% argument.  Spaces before the \cs{bigstrut} are generally ignored (by a
% single \cs{unskip}).
%
% \DescribeMacro{\bigstrutjot}
% Note: The \Package{multirow} package makes use of \cs{bigstrutjot}.  If
% both packages are used, they can be used in either order, as each checks
% to see if the other has already defined \cs{bigstrutjot}.  However,
% the default values they set are different: if only \Package{multirow}
% is used, \cs{bigstrutjot} will be set to \texttt{3pt}.
% If \Package{bigstrut} is used, with or without \Package{multirow},
% \cs{bigstrutjot} will be \texttt{2pt}.
%
% \section{Using \Package{bigdelim}}
%
% The package is for working in a \env{tabular} or \env{array}
% environment, in which the \Package{multirow} package is also used.
%
% \DescribeMacro{\ldelim}
% \DescribeMacro{\rdelim}
% Syntax of use is
% \begin{quote}
%   \cs{ldelim} ( \marg{n}\oarg{vmove}\marg{width}\oarg{text}\\
%   \cs{rdelim} ) \marg{n}\oarg{vmove}\marg{width}\oarg{text}
% \end{quote}
% The commands are used in a column of a \env{tabular} or
% \env{array}; they create a big parenthesis, brace or whatever
% delimiter that extends over the \meta{n} rows starting at the one
% containing the command.  Corresponding cells in the following rows
% must be explicitly given as empty cells.
%
% The first parameter is a delimiter to be used, e.g., \cs{\char`\{}
% \cs{\char`\}} \texttt{[} \texttt{]} \texttt{(} \texttt{)}\,---\,in fact,
% anything that can be used with \cs{left} or \cs{right}, as appropriate.
%
% Here is an example:
% \begin{quote}
% \begin{verbatim}
% \begin{equation}
%   \begin{array}{ccccccc}
%     \ldelim({4}{4mm} & x & x & x & x & \rdelim){4}{4mm}  \\
%     & x & x & x & x &  & i \\
%     & x & x & x & x &  & j \\
%     & x & x & x & x &  \\
%     &   & u & v &   &
%   \end{array}
% \end{equation}
% \end{verbatim}
% \end{quote}
%
% \begin{quote}
% \begin{equation}
%   \begin{array}{ccccccc}
%     \ldelim({4}{4mm} & x & x & x & x & \rdelim){4}{4mm}  \\
%     & x & x & x & x &  & i \\
%     & x & x & x & x &  & j \\
%     & x & x & x & x &  \\
%     &   & u & v &   &
%   \end{array}
% \end{equation}
% \end{quote}
%
% The optional parameter \meta{vmove} is a length used for fine-tuning: the
% delimiter (with the optional \meta{text}) will be raised (or lowered, if
% \meta{vmove} is negative) by that length above (below) wherever it would
% otherwise have gone. This is just like with \cs{multirow}, but note that
% here the \meta{vmove} goes before the \meta{width}.
%
% When \cs{ldelim} is used, the optional \meta{text} is set centred to the left of \cs{ldelim}.
% If \cs{rdelim} is used it is set to the right of \cs{rdelim}.  The
% \meta{width} parameter is the space that is reserved for
% the delimiter and its text; as with the
% \Package{multirow} package, the \meta{width} may be given as
% \texttt{*}. Compare for example these:
% \begin{quote}
% \begin{minipage}{0.6\linewidth}
% \begin{verbatim}
% \begin{tabular}{p{2em}l}
%   \ldelim\{{3}{*}[type] & dvi \\
%                         & ps \\
%                         & pdf \\
% \end{tabular}
% \end{verbatim}
% \end{minipage}
% \begin{minipage}{0.35\linewidth}
% \begin{tabular}{p{2em}l}
%   \ldelim\{{3}{*}[type] & dvi \\
%                             & ps \\
%                             & pdf \\
%    \end{tabular}
%    \\[2ex]
% \end{minipage}
%
% \begin{minipage}{0.6\linewidth}
% \begin{verbatim}
% \begin{tabular}{l@{\,}l}
%   \ldelim\{{3}{*}[type] & dvi \\
%                         & ps \\
%                         & pdf \\
% \end{tabular}
% \end{verbatim}
% \end{minipage}
% \begin{minipage}{0.35\linewidth}
%   \begin{tabular}{l@{\,}l}
%   \ldelim\{{3}{*}[type] & dvi \\
%                          & ps \\
%                          & pdf \\
%   \end{tabular}
%   \\[2ex]
% \end{minipage}
% \end{quote}
% In the first example we cheated: by using a column width that is too
% small, we swallowed up some of the intercolumn space, at the cost of
% an ``Overfull \cs{hbox}'' message. In the second example we did it the
% proper way by inserting \verb+@{\,}+ to replace the default
% intercolumn space with a narrow space.
%
% Also the commands may be
% used in the last row of the extension with a negative \meta{n}
% \index{colortbl=\Package{colortbl}|usage}
% argument.  This is useful in combination with the \Package{colortbl}
% \index{xcolor=\Package{xcolor}|usage}
% or \Package{xcolor} packages
% (see the discussion in section \ref{sec:multirow} on \Package{multirow}).
% If there are unusually tall rows you may have to enlarge \meta{n}
% (you can use fractional values).
% If you have horizontal lines that interact with the braces you are
% \index{hhline=\Package{hhline}|usage}
% advised to use the \Package{hhline} package to make the lines.
%
% If you decrease or eliminate the intercolumn space with \verb|@{}| and
% use colored backgrounds with the \Package{colortbl} or
% \Package{xcolor} packages (commands \verb|\columncolor|,
% \verb|\rowcolor|, \verb|\rowcolors| or \verb|\cellcolor|),
% \index{columncolor=\verb!*+\columncolor+|usage}
% \index{rowcolor=\verb!*+\rowcolor+|usage}
% \index{rowcolors=\verb!*+\rowcolors+|usage}
% \index{cellcolor=\verb!*+\cellcolor+|usage}
% you will notice that part of the brace will be cut off. In reality it
% will be overwritten with the color of the next cell. See this example:
%
% \begin{quote}
% \begin{minipage}{0.6\linewidth}
% \begin{verbatim}
% \rowcolors{2}{green!25}{green!75}
% \begin{tabular}{ c @{}c c c }
%                  & 1 & 2 & 3 \\
%                  & 4 & 5 & 6 \\
% \ldelim\{{-3}{*} & 7 & 8 & 9 \\
% \end{tabular}
% \end{verbatim}
% \end{minipage}
% \begin{minipage}{0.35\linewidth}
% \rowcolors{2}{green!25}{green!75}
% \begin{tabular}{ c @{}c c c }
%                  & 1 & 2 & 3 \\
%                  & 4 & 5 & 6 \\
% \ldelim\{{-3}{*} & 7 & 8 & 9 \\
% \end{tabular}
%    \\[2ex]
% \end{minipage}
% \end{quote}
%
% This is not a problem of \Package{multirow} or \Package{bigdelim};
% it will also happen if there is normal text in the column before the
% \verb|@{}|. The reason is that these color commands extend the color
% to cover the intercolumn spaces on both sides to prevent gaps in the
% color. The size of these so-called \textit{overhangs} is \verb|\tabcolsep| (or
% \verb|\arraycolsep| when an \env{array} is used) on each side.
% However, when \verb|@{}| is used there is no such intercolumn space
% and the extension covers parts of the previous column.
% This can be cured by setting the left \textit{overhang} explicitly to
% 0pt with a \verb|\columncolor| command in the tabular header, like
% \verb|>{\columncolor{white}[0pt][\tabcolsep]}|. Unfortunately the
% explicit color \texttt{white}, removes the transparency of the column,
% so if there are cells in that column that have no explicit color,
% these cells are affected. If the background of the \env{tabular} is
% white, this normally will not be noticed, but if the background color
% is changed, for example with the \verb|\pagecolor| command, then that color
% should be used rather than \texttt{white}. Unfortunately, there is no
% command to specify the \textit{overhangs} without also specifying a
% color.
%
% In the following example we have done this. In order to keep the table
% header compact, we put the deinition in a \verb|\newcolumn| command
% (using the \Package{array} package).
%
% \begin{quote}
% \begin{minipage}{0.6\linewidth}
% \begin{verbatim}
% \newcolumntype{z}{@{}>{\columncolor{white}[0pt][\tabcolsep]}}
% \rowcolors{2}{green!25}{green!75}
% \begin{tabular}{ c zc c c }
%                  & 1 & 2 & 3 \\
%                  & 4 & 5 & 6 \\
% \ldelim\{{-3}{*} & 7 & 8 & 9 \\
% \end{tabular}
% \end{verbatim}
% \end{minipage}
% \begin{minipage}{0.35\linewidth}
% \newcolumntype{z}{@{}>{\columncolor{white}[0pt][\tabcolsep]}}
% \rowcolors{2}{green!25}{green!75}
% \begin{tabular}{ c zc c c }
%                  & 1 & 2 & 3 \\
%                  & 4 & 5 & 6 \\
% \ldelim\{{-3}{*} & 7 & 8 & 9 \\
% \end{tabular}
%    \\[2ex]
% \end{minipage}
% \end{quote}
%
% In case you want to have a paragraph type text as optional parameter you could put it in a \cs{parbox}.
% Alternatively you could add an extra column with the text in a \cs{multirow}, like in
% \begin{quote}
% \begin{verbatim}
% \begin{tabular}{l@{}l@{}l}
%    dvi   & \rdelim\}{3}{1em} & \multirow{3}{4cm}{These are the output types,
%         that are commonly used for \TeX.} \\
%    ps    & & \\
%    pdf   & & \\
% \end{tabular}
% \end{verbatim}
% \end{quote}
% \begin{quote}
% \begin{tabular}{l@{}l@{}l}
%    dvi   & \rdelim\}{3}{1em} & \multirow{3}{4cm}{These are the output types,
%         that are commonly used for \TeX.} \\
%    ps   & & \\
%    pdf   & & \\
% \end{tabular}
% \end{quote}
% Note that we used \texttt{@\{\}} to eliminate the intercolumn space to
% get the text tight to the brace.
% \section{Contact Information}
%
% Pieter van Oostrum\\
% email: \texttt{pieter@vanoostrum.org}\\
% www: \url{http://pieter.vanoostrum.org}
% \\[1ex]
% The source code can be found on Github:\\
% \url{https://github.com/pietvo/multirow}\\
% Bugs can be reported at\\
% \url{https://github.com/pietvo/multirow/issues}
%
% \StopEventually{%
% \PrintChanges
% \PrintIndex}
%
% \section{Implementation}
%
%
%\subsection{The \Package{multirow} package}
%\label{sec:pack-multirow}
% \iffalse
%<*multirow>
% \fi
%
% \changes{multirow v1.0}{}{distributed anonymously, based on a Usenet posting}
% \changes{multirow v1.1}{}{allow it to work without bigstrut.sty (Pieter van Oostrum)}
% \changes{multirow v1.2}{}{modified by Jerry Leichter for the same goal, but using a
%   different approach which will work properly with bigstrut.sty}
% \changes{multirow v1.2a}{}{modified by Pieter van Oostrum to use \cs{vskip}
%   instead of \cs{raise} in positioning, avoiding making rows too high
%   when the adjustment is large}
% \changes{multirow v1.3}{}{modified by Pieter van Oostrum to work properly in a p{} column
%    (\cs{leavevmode} added)}
% \changes{multirow v1.4}{}{modified by Pieter van Oostrum to check for the special case that
%       the width is given as an *. In this case the natural
%       width of the text argument will be used and the argument
%       is processed in LR-mode.}
% \changes{multirow v1.5}{}{modified by Pieter van Oostrum: Added a \texttt{\%} after \texttt{\cs{hbox}\{\#5\}\cs{vfill}}.\\
%       Added \cs{strut}s around \#5 for better vertical positioning.
%       Additional coding for negative value of \meta{nrows}.}
% \changes{multirow v1.6}{2004/05/05}{modified by Pieter van Oostrum: Replace a space by \cs{relax} after
%       \texttt{\cs{advance}\cs{multirow@dima}\#4}}
%
% \changes{v1.9a}{2016/09/23}{Implement the \texttt{debug} option.}
%
% \begin{macro}{\ifmultirowdebug}
% \begin{macro}{\multirowdebugtrue}
% \begin{macro}{\multirowdebugfalse}
% This is a boolean to [de]activate debugging (showing the generated box
% contents). It is activated by the \texttt{debug} package option. The
% \cs{newif} initializes it to false.
%    \begin{macrocode}
\newif\ifmultirowdebug
\DeclareOption{debug}{\multirowdebugtrue}
%    \end{macrocode}
% \end{macro}
% \end{macro}
% \end{macro}
%
% \changes{v1.9b}{2016/09/24}{Implement the \texttt{longtable} option.}
% \changes{v2.5}{2019/05/31}{Make the redefinition of \cs{@cline}
% compatible with the \Package{colortbl} package.}
% \begin{macro}{\cline}
% The package option \texttt{longtable} redefines the \cs{cline} macro to
% work around a bug in \Package{longtable}. See section \ref{sec:use-with-longtable}\footnote{%
% Thanks to David Carlisle. See
% \href{http://tex.stackexchange.com/questions/52100/longtable-multirow-problem-with-cline-and-nopagebreak\#answer-52101}{his answer on stackexchange}.}.
% First we check if the macro \cs{CT@arc} is defined. If so, this
% indocates that the \Package{colortbl} package is loaded. As
% \Package{colortbl} also redefines \cs{@cline}, we must take this into
% account with our own redefinition of \cs{@cline}.
%    \begin{macrocode}
\DeclareOption{longtable}{%
\AtBeginDocument{%
\@ifundefined{CT@arc}
{\def\@cline#1-#2\@nil{%
  \omit
  \@multicnt#1%
  \advance\@multispan\m@ne
  \ifnum\@multicnt=\@ne\@firstofone{&\omit}\fi
  \@multicnt#2%
  \advance\@multicnt-#1%
  \advance\@multispan\@ne
  \leaders\hrule\@height\arrayrulewidth\hfill
  \cr
  \noalign{\nobreak\vskip-\arrayrulewidth}}}
{\def\@cline#1-#2\@nil{%
  \omit
  \@multicnt#1%
  \advance\@multispan\m@ne
  \ifnum\@multicnt=\@ne\@firstofone{&\omit}\fi
  \@multicnt#2%
  \advance\@multicnt-#1%
  \advance\@multispan\@ne
  {\CT@arc@\leaders\hrule\@height\arrayrulewidth\hfill}%
  \cr
  \noalign{\nobreak\vskip-\arrayrulewidth}}}
}}
%    \end{macrocode}
% \end{macro}
%
% \changes{v1.9b}{2016/09/26}{Implement the \texttt{supertabular} option
% and the \cs{STneed} command.}
% The package option \texttt{supertabular} redefines \verb+\\*+ inside a
% \env{supertabular}.
% The redefinition is delayed until the \verb+\begin{document}+.
%
% \changes{v2.6}{2020/12/26}{Adapt the definition to be compatible with
% modern versions of \texttt{supertabular}}
% \texttt{Supertabular} version 4.1f and later need a call to
% \cs{ST@save@lineno} to function properly, but earlier versions cannot
% use this as it doesn't exist in these versions. So the definitions of
% both \cs{ST@tabularcr} and \cs{MRST@cr} are different depending on whether
% \cs{ST@save@lineno} is defined. There are also some other differences,
% so some intermediate versions of \texttt{supertabular} might need more
% subtle adaptations, but for now we leave it at that.
%
%    \begin{macrocode}
\DeclareOption{supertabular}{%
\AtBeginDocument{%
%    \end{macrocode}
% \begin{macro}{\ST@tabularcr}
% This macro is the definition of \verb+\\+ inside a
% \env{supertabular}.  We check for a \verb+*+, and if it is
% present we call our own version, otherwise the
% \env{supertabular} version. First we get the older version for
% pre-4.1f \texttt{supertabular}, then the newer version.
%    \begin{macrocode}
\ifx\ST@save@lineno\undefined
\def\ST@tabularcr{%
  {\ifnum0=`}\fi
  \@ifstar{\MRST@xtabularcr}{\ST@xtabularcr}}
%    \end{macrocode}
% \end{macro}
% \begin{macro}{\MRST@cr}
% \cs{MRST@cr} is a truncated copy of \cs{ST@cr}. It does all the bookkeeping
% about the space the \env{longtable} occupies, but it doesn't
% do the pagebreaking part.
%    \begin{macrocode}
\def\MRST@cr{%
  \noalign{%
    \ifnum\ST@pboxht<\ST@lineht
      \global\advance\ST@pageleft -\ST@lineht
      \global\ST@prevht\ST@lineht
    \else
      \global\advance\ST@pageleft -\ST@pboxht
      \global\advance\ST@pageleft -0.1\ST@pboxht
      \global\advance\ST@pageleft -\ST@stretchht
      \global\ST@prevht\ST@pboxht
      \global\ST@pboxht\z@
    \fi
    \global\advance\ST@pageleft -\ST@toadd
    \global\ST@toadd=\z@}}
\else
%    \end{macrocode}
% These are the newer versions.
%    \begin{macrocode}
\def\ST@tabularcr{%
  {\ifnum0=`}\fi
  \ST@save@lineno
  \@ifstar{\MRST@xtabularcr}{\ST@xtabularcr}}
\def\MRST@cr{%
  \noalign{%
    \ifnum\ST@pboxht<\ST@lineht
      \global\advance\ST@pageleft -\ST@lineht
      \global\ST@prevht\ST@lineht
    \else
     \ST@trace@cr\thr@@{Added par box with height \the\ST@pboxht}%
      \global\advance\ST@pageleft -\ST@pboxht
      \global\advance\ST@pageleft -0.1\ST@pboxht
      \global\ST@prevht\ST@pboxht
      \global\ST@pboxht\z@
    \fi
    \global\advance\ST@pageleft -\ST@toadd
    \global\ST@toadd=\z@
    \ST@trace@cr\thr@@{Space left for tabular: \the\ST@pageleft}}}
\fi
%    \end{macrocode}
% \end{macro}
% \begin{macro}{\MRST@xtabularc}
% \begin{macro}{\MRST@argtabularc}
% \begin{macro}{\MRST@xargtabularc}
% \begin{macro}{\MRST@yargtabularc}
% These are copies of the corresponding macros from
% \Package{supertabular}, but instead of \cs{ST@cr} they call \cs{MRST@cr}.
%    \begin{macrocode}
\def\MRST@xtabularcr{%
  \@ifnextchar[%]
    {\MRST@argtabularcr}%
    {\ifnum0=`{\fi}\cr\MRST@cr}}
\def\MRST@argtabularcr[#1]{%
  \ifnum0=`{\fi}%
  \ifdim #1>\z@
    \unskip\MRST@xargarraycr{#1}
  \else
    \MRST@yargarraycr{#1}%
  \fi}
\def\MRST@xargarraycr#1{%
  \@tempdima #1\advance\@tempdima \dp \@arstrutbox
  \vrule \@height\z@ \@depth\@tempdima \@width\z@ \cr
  \noalign{\global\ST@toadd=#1}\MRST@cr}
\def\MRST@yargarraycr#1{%
  \cr\noalign{\vskip #1\global\MRST@toadd=#1}\MRST@cr}
}
%    \end{macrocode}
% \end{macro}
% \end{macro}
% \end{macro}
% \end{macro}
% \begin{macro}{\STneed}
% This macro can be used in a \env{supertabular} to indicate how
% much space a multirow entry needs. See section~\ref{sec:use-with-supertabular}.
%    \begin{macrocode}
\def\STneed#1{\ifdim\ST@pageleft<#1\ST@newpage\ST@next\fi}
}
%    \end{macrocode}
% \end{macro}
%    \begin{macrocode}
\ProcessOptions
%    \end{macrocode}
%
% \cs{multirow@colwidth} is a length that is used to implement the ``\texttt{=}'' variant of \meta{width}.
%
% \changes{v1.9}{2016/09/15}{Implement the ``\texttt{!=}'' option for \cs{multirow}'s \meta{width} parameter.}
% \begin{macro}{\multirow@colwidth}
% \cs{multirow@colwidth} is a length that is used to implement the ``\texttt{=}'' variant of \meta{width}.
%    \begin{macrocode}
\newlength{\multirow@colwidth}
%    \end{macrocode}
% \end{macro}
%
% \changes{v1.9}{2016/09/15}{Give \Package{multirow} its own temp
% registers, so that we can safely pass the box height to \Package{bigdelim}.}
% \changes{v2.2}{2016/11/25}{Eliminate \cs{multirow@cnta.}}
% \begin{macro}{\multirow@cntb}
% \begin{macro}{\multirow@dima}
%   Define two counters and a length for internal use in \cs{multirow}.
%    \begin{macrocode}
\newcount\multirow@cntb
\newlength\multirow@dima
%    \end{macrocode}
% \end{macro}
% \end{macro}
% \begin{macro}{\multirow@setcolwidth}
% This macro calculates \cs{multirow@colwidth} for an entry that has the
% \meta{width} given as ``\texttt{=}''. We check if we are inside a
% \env{tabulary} environment, by checking if \cs{TY@final} is defined.
% If not, then \cs{multirow@colwidth} = \cs{hsize}. The \env{tabulary}
% environment will make two passes. On the first pass, we set \cs{multirow@colwidth}
% to the size that the text would have in LR mode (with newlines replaced
% by spaces), so that \env{tabulary} will gives us enough space.
% On the second pass (characterized by \cs{TY@box} = \cs{TY@box@v}) we use
% the value that \env{tabulary} has given us in \cs{hsize}.
% This algorithm is not perfect, but good enough in most cases.
% \changes{v2.7}{2021/01/29}{Make \cs{multirow@setcolwidth} \cs{long} to allow
% multi-paragraph text}
%    \begin{macrocode}
\long\def\multirow@setcolwidth#1{%
    \ifx\TY@final\multirow@undefined \multirow@colwidth=\hsize
    \else
      \ifx\TY@box\TY@box@v\multirow@colwidth=\hsize
      \else \setbox0\hbox
        {\let\\\space\let\newline\space #1}\multirow@colwidth=\wd0
      \fi
    \fi}
%    \end{macrocode}
% \end{macro}
%
% \begin{macro}{\multirowsetup}
% \cs{multirowsetup} is executed at the beginning of each \cs{multirow}.
%    \begin{macrocode}
\newcommand\multirowsetup{\raggedright}
%    \end{macrocode}
% \end{macro}
%
% \begin{macro}{\multirow@vbox}
% This creates the \cs{vbox}. Parameters:\\
% \texttt{\#1} = \meta{vpos}, \texttt{\#2} = initialization code (for
% example to set the width of the \cs{parbox}), \texttt{\#3} = box contents.\\
% Depending on the \meta{vpos} parameter, it will be top-aligned,
% vertically centered, or bottom-aligned. This is done by inserting
% \cs{vfill} in the proper places.\\
% Note: the \cs{relax} is to protect against an empty \meta{vpos}
% argument.
%    \begin{macrocode}
\long\def\multirow@vbox#1#2#3{\setbox0\vtop to \multirow@dima{#2%
    \if #1t\relax\else\vfill\fi
    \multirowsetup #3\if #1b\relax\else\vfill\fi}}
%    \end{macrocode}
% \end{macro}
%
% \begin{macro}{\multirow}
% Make an entry that will span multiple rows of a table.\\
% First collect all the arguments and replace missing optional arguments by their default values.
% \changes{v1.8}{2016/09/14}{Add the optional first parameter \meta{vpos}.}
%    \begin{macrocode}
%% \multirow [vpos] {nrows} [bigstruts] {width} [vmove] {text}
\newcommand\multirow[2][c]{\@multirow[#1]{#2}}
\def\@multirow[#1]#2{\@ifnextchar[{\@@multirow[#1]#2}{\@@multirow[#1]#2[0]}}
\def\@@multirow[#1]#2[#3]#4{\@ifnextchar[{\@xmultirow[#1]{#2}[#3]{#4}}%
                                {\@xmultirow[#1]{#2}[#3]{#4}[0pt]}}
%    \end{macrocode}
% \changes{v1.9a}{2016/09/23}{Add the optional prefix to the \meta{bigstruts} parameter.}
% \begin{macro}{\multirow@piii}
% \begin{macro}{\ifmultirow@prefixt}
% \begin{macro}{\multirow@prefixttrue}
% \begin{macro}{\multirow@prefixtfalse}
% \begin{macro}{\ifmultirow@prefixb}
% \begin{macro}{\multirow@prefixbtrue}
% \begin{macro}{\multirow@prefixbfalse}
% This macro splits off a \texttt{t}, \texttt{b}, or \texttt{tb} prefix
% of the \meta{bigstruts} argument, and sets \cs{multirow@cntb} to
% the numerical value. The prefix is remembered in two booleans:
% \cs{ifmultirow@prefixt} and \cs{ifmultirow@prefixb}.
%    \begin{macrocode}
\newif\ifmultirow@prefixt
\newif\ifmultirow@prefixb
\def\multirow@piii#1#2#3\end{\multirow@prefixtfalse\multirow@prefixbfalse
  \if t#1\multirow@prefixttrue
    \if b#2\multirow@prefixbtrue \multirow@cntb=#3%
    \else \multirow@cntb=#2#3%
    \fi
  \else
    \if b#1\multirow@prefixbtrue \multirow@cntb=#2#3%
    \else \multirow@cntb=#1#2#3%
    \fi
  \fi}
%    \end{macrocode}
% \end{macro}
% \end{macro}
% \end{macro}
% \end{macro}
% \end{macro}
% \end{macro}
% \end{macro}
%
% This is the real workhorse. It starts with splitting the
% \meta{bigstruts} argument, and then calculating the height of the multirow box.
% \changes{v2.2}{2016/11/25}{Support fractional values for \meta{nrows}.}
% Because \meta{nrows} (\verb+#2+) can be fractional, we cannot use
% \cs{ifnum} to test for positive or negative. Therefore we use
% \cs{ifdim} by putting a unit (\texttt{pt}) after the number.
% \changes{v2.4}{2019/01/01}{Support calc compatible expressions for  \meta{width} and
% \meta{vmove}.}
% \changes{v2.7}{2021/01/29}{Make \cs{@xmultirow} \cs{long} to allow
% multi-paragraph text}
%    \begin{macrocode}
\long\def\@xmultirow[#1]#2[#3]#4[#5]#6{%
  \expandafter\multirow@piii#3\relax\end%
  \setlength\multirow@dima{#2\ht\@arstrutbox}%
  \addtolength\multirow@dima{#2\dp\@arstrutbox}%
  \ifdim#2pt<\z@\setlength\multirow@dima{-\multirow@dima}\fi
  \addtolength\multirow@dima{\multirow@cntb\bigstrutjot}%
%    \end{macrocode}
% \changes{v1.9a}{2016/09/23}{Redo the \cs{vbox} calculation and positioning.}
%  The text is set in a \cs{vbox} by calling \cs{multirow@vbox}. \\
% If the \meta{width} argument is \texttt{*} set just the text in the \cs{vbox}.
%    \begin{macrocode}
  \if*#4\multirow@vbox{#1}{}{\hbox{\strut#6\strut}}%
%    \end{macrocode}
% Otherwise set it in a \cs{parbox} inside a \cs{vbox}. \\
% If the \meta{width} argument is given as ``\texttt{=}'', we calculate
% \cs{multirow@colwidth} and use that as width of the \cs{parbox}.
%    \begin{macrocode}
  \else \if=#4\multirow@setcolwidth{#6}%
    \multirow@vbox{#1}{\setlength\hsize{\multirow@colwidth}\@parboxrestore}{\strut#6\strut\par}%
%    \end{macrocode}
% Otherwise the given argument is used as the width of the \cs{parbox}.
%    \begin{macrocode}
  \else \multirow@vbox{#1}{\setlength\hsize{#4}\@parboxrestore}{\strut#6\strut\par}%
  \fi \fi
%    \end{macrocode}
% Now position the \cs{vbox} properly. More details are given in the
% appendix. The overview of the calculation of the shift amount can be
% found in section~\ref{sec:overview}.
%
% If \meta{nrows}${}> 0$: \\
% If \meta{vpos} = \texttt{[t]}, then the box is already positioned
% correctly (the baseline is on the baseline of the row). However, later
% the top of the box will be taken as the reference point (instead
% of the baseline), therefore we take
% the height of the box (h) as the shift amount.  See fig.~\ref{fig:top-t}.\\
% If \meta{vpos} = \texttt{[c]} we shift it up h1 (see fig.~\ref{fig:top-c}), where h1 = \cs{ht}\cs{@arstrutbox}
% + (\cs{bigstrutjot} \cs{ifmultirow@prefixt}).\\
% If \meta{vpos} = \texttt{[b]} we shift it up h1 + h2 (see fig.~\ref{fig:top-b}), where h2 = \cs{dp}\cs{@arstrutbox}
% + (\cs{bigstrutjot} \cs{ifmultirow@prefixb}).\\
% We calculate the required shift  in \cs{multirow@dima}.
%    \begin{macrocode}
  \ifdim#2pt>\z@
    \if#1t\relax\setlength\multirow@dima{\ht0}\else
      \setlength\multirow@dima{\ht\@arstrutbox}%
      \ifmultirow@prefixt \addtolength\multirow@dima{\bigstrutjot}\fi
      \if#1b\relax \addtolength\multirow@dima{\dp\@arstrutbox}%
        \ifmultirow@prefixb \addtolength\multirow@dima{\bigstrutjot}\fi
      \fi
    \fi
%    \end{macrocode}
% If \meta{nrows}${}< 0$: \\
% If \meta{vpos} = \texttt{[t]}, shift the box up $\textrm{H} -
% \textrm{h1} -\textrm{h2} + \textrm{h}$.  See fig.~\ref{fig:bot-t}.\\
% If \meta{vpos} = \texttt{[c]}, shift the box up $\textrm{H} - \textrm{h2}$.  See fig.~\ref{fig:bot-c}.\\
% If \meta{vpos} = \texttt{[b]}, shift the box up H.  See fig.~\ref{fig:bot-b}.\\
% H is the current value of \cs{multirow@dima}.
%    \begin{macrocode}
  \else
    \if#1b\relax\else
      \addtolength\multirow@dima{-\dp\@arstrutbox}%
      \ifmultirow@prefixb \addtolength\multirow@dima{-\bigstrutjot}\fi
      \if#1t\relax\addtolength\multirow@dima{-\ht\@arstrutbox}%
        \ifmultirow@prefixt \addtolength\multirow@dima{-\bigstrutjot}\fi
        \addtolength\multirow@dima{\ht0}%
      \fi
    \fi
  \fi
%    \end{macrocode}
% \changes{v2.1}{2016/10/11}{Set depth of final \cs{vbox} to 0, to
% prevent a tall multirow line to push the following rows downwards.}
% Finally, we add the \meta{vmove} argument (\verb+#5+), and go into
% horizontal mode. Then we shift the box up by putting a \cs{vskip} above it,
% and add it to the output. Because of the \cs{vskip} the resulting box
% will have a height 0.  We set the depth of the \cs{vbox}
% to 0, so that it will not influence the depth of the current row.\\
% If \cs{multirowdebug} is true, we show the box.
%    \begin{macrocode}
  \addtolength\multirow@dima{#5}%
  \leavevmode
  \setbox0\vtop{\vskip-\multirow@dima\box0\vss}\dp0=\z@
  \ifmultirowdebug{\showboxdepth=5 \showboxbreadth=10 \showbox0}\fi
  \box0
}
%    \end{macrocode}
% \end{macro}
% \begin{macro}{\bigstrutjot}
% Define \cs{bigstrutjot} if not already defined.
%    \begin{macrocode}
\@ifundefined{bigstrutjot}{\newdimen\bigstrutjot \bigstrutjot=\jot}{}
%    \end{macrocode}
% \end{macro}
% \iffalse
%</multirow>
% \fi
%
% \subsection{The \Package{bigstrut} package}
% \label{sec:pack-bigstrut}
%
% \iffalse
%<*bigstrut>
% \fi
% \changes{bigstrut v1.0}{1994/05/31}{Initial version}
% \begin{macro}{\bigstrutjot}
% This is a length. By default it is set to 2pt. You can change it
% with the \cs{setlength} command.
%    \begin{macrocode}
\@ifundefined{bigstrutjot}{\newdimen\bigstrutjot}{}\bigstrutjot=2pt
%    \end{macrocode}
% \end{macro}
%
% \changes{bigstrut v2.4}{2018/12/30}{Add \cs{leavevmode}
% at the beginning to force horizontal mode}
% \begin{macro}{\bigstrut}
% This macro inserts a strut. Depending on the optional parameter it extends
% above and/or below the standard \env{array}/\env{tabular} strut.
%    \begin{macrocode}
\newcommand\bigstrut[1][x]{%
  \leavevmode\unskip\@tempdima=\ht\@arstrutbox \@tempdimb=\dp\@arstrutbox
  \ifx #1b\relax \else \advance\@tempdima by \bigstrutjot\fi
  \ifx #1t\relax \else \advance\@tempdimb by \bigstrutjot\fi
  \hbox{\vrule \@height\@tempdima \@depth\@tempdimb \@width\z@}\ignorespaces}
%    \end{macrocode}
% \end{macro}
%
% \iffalse
%</bigstrut>
% \fi
%
% \subsection{The \Package{bigdelim} package}
% \label{sec:pack-bigdelim}
% \iffalse
%<*bigdelim>
% \fi
% \changes{bigdelim v0.0}{1994/10/02}{bigbrace.sty by \O ystein Bache}
% \changes{bigdelim v1.0}{1999/11/05}{Initial version bigdelim.sty}
%    \begin{macrocode}
\RequirePackage{multirow}
%    \end{macrocode}
% \begin{macro}{\ldelim}
% This macro typesets a left delimiter. It calls \cs{multirow} with the proper arguments.
% The size of the delimiter is determined by putting a \cs{vbox} with the proper height and
% zero width next to it. The height is the one that \cs{multirow}
% already has calculated in \cs{multirow@dima}. That calculation uses
% the size of \cs{@arstrutbox}, which is set by \texttt{tabular} or
% \texttt{array} environments. In case it is not set, we initialize it
% to a default value.
% \changes{bigdelim v2.3}{2018/08/03}{Replace \cs{textrm} by \cs{textnormal}}
% \changes{bigdelim v2.6}{2020/12/26}{Initialize \cs{@arstrutbox} if not defined}
% \changes{bigdelim v2.8}{2021/03/15}{Add optional argument \meta{vmove}}
%    \begin{macrocode}
\newcommand\ldelim[2]{\@ifnextchar[{\@ldelim{#1}{#2}}{\@ldelim{#1}{#2}[0pt]}}
\def\@ldelim#1#2[#3]#4{\@ifnextchar[{\@@ldelim{#1}{#2}{#3}{#4}}{\@@ldelim{#1}{#2}{#3}{#4}[\null]}}
\def\@@ldelim#1#2#3#4[#5]%
 {\ifvoid\@arstrutbox\setbox\@arstrutbox\hbox{\strut}\fi
  \multirow{#2}{#4}[#3]{%
   \ensuremath
     {\left.\vcenter{\hsize=0pt\vrule height \multirow@dima width 0pt}%
     \textnormal{#5}\right#1}}}
%    \end{macrocode}
% \end{macro}
%
% \begin{macro}{\rdelim}
% This macro typesets a right delimiter. It calls \cs{multirow} with the
% proper arguments, similar to \cs{ldelim}.
% \changes{bigdelim v2.3}{2018/08/03}{Replace \cs{textrm} by \cs{textnormal}}
% \changes{bigdelim v2.6}{2020/12/26}{Initialize \cs{@arstrutbox} if not defined}
% \changes{bigdelim v2.8}{2021/03/15}{Add optional argument \meta{vmove}}
%    \begin{macrocode}
\newcommand\rdelim[2]{\@ifnextchar[{\@rdelim{#1}{#2}}{\@rdelim{#1}{#2}[0pt]}}
\def\@rdelim#1#2[#3]#4{\@ifnextchar[{\@@rdelim{#1}{#2}{#3}{#4}}{\@@rdelim{#1}{#2}{#3}{#4}[\null]}}
\def\@@rdelim#1#2#3#4[#5]%
 {\ifvoid\@arstrutbox\setbox\@arstrutbox\hbox{\strut}\fi
  \multirow{#2}{#4}[#3]{%
   \ensuremath
     {\left#1\vcenter{\hsize=0pt\vrule height \multirow@dima width 0pt}%
       \textnormal{#5}\right.}}}
%    \end{macrocode}
% \end{macro}
% \iffalse
%</bigdelim>
% \fi
%
% \appendix
%
% \section{Appendix}
% \label{sec:appendix}
%
% \index{vbox=\verb!*+\vbox+|usage}
%
% In this section we explain the \cs{vbox} positioning in \cs{multirow}.
% The positioning depends on the \meta{nrows}, \meta{vpos},
% \meta{bigstruts} and \meta{vmove} arguments. The box is constructed with \cs{vtop}. The
% algorithm of \cs{vtop} is described in \emph{The \TeX{}book}, p.~81.
%
% Each case is described by a figure. In the figure the lefthand column
% indicates the context of the tabular in which the multirow appears,
% i.e \meta{nrows} rows. The righthand column is the multirow box that
% is to be inserted. The baseline is the natural position where the material will be
% positioned in the first place. Later it will be shifted up
% to the desired location.
%
% H is the calculated height of the box: $\meta{nrows} \times
% \textrm{the natural height of a row}  + \meta{bigstruts} \times
% \mathrm{\cs{bigstrutjot}} $.
%
% {topstrut} = \cs{bigstrutjot} if there is  a \cs{bigstrut} on the top of
% the first row (as indicated by the \texttt{t} prefix in the
% \meta{bigstruts} argument), otherwise 0.
%
% {botstrut} = \cs{bigstrutjot} if there is  a \cs{bigstrut} on bottom of
% the last row (as indicated by the \texttt{b} prefix in the
% \meta{bigstruts} argument), otherwise 0.
%
% h1 = $\textrm{height of a tabular row} + \textrm{topstrut}$
%
% h2 = $\textrm{depth of a tabular row} + \textrm{botstrut}$
%
% Note: the following descriptions describe the vertical shift of the
% box without taking the \meta{vmove} into account. In all cases
% \meta{vmove} has to be added if it is given.
%
% \subsection{Case \meta{nrows}${}> 0$}
%
% \subsubsection*{\meta{vpos}  = [t]}
%
% In this case the \cs{vbox} contains the text followed by a \cs{vfill}.
% Such a \cs{vbox} has a height that is the height of the top line of
% the text (h). H = height + depth of the box. This means that the box is
% already positioned correctly. However, later we will put the box
% inside another \cs{vbox}, with a \cs{vskip} on to of it, and this will
% make the top of the box its reference point. Therefore we will
% have to shift it up again over a distance h
% (which probably will be different from the height of the tabular row).
% So the total shift becomes h. See fig.~\ref{fig:top-t}.
%
% Alternatively, we could have omitted the \cs{vskip} in this case,
% thereby leaving the baseline undisturbed, but this would make the code
% unsymmetrical. Moreover, this would not work when a non-zero
% \meta{vmove} is present. Therefore we choose to set the shift amount to h here.
%
% \begin{figure}[htpb]
%   \centering
% \begin{tikzpicture}
%   \draw (2, -0.5) rectangle (5,4.5);
%   \foreach \y in {0,...,4} \draw (2, \y) -- (5, \y);
%   \draw (0, 4.2) node (topstrut) {topstrut};
%   \draw[->] (topstrut) -- (2, 4.2);
%   \draw (0, -0.25) node (botstrut) {botstrut};
%   \draw[->] (botstrut) -- (2, -0.25);
%   \draw[<->] (5.3, -0.5) -- (5.3, 4.5);
%   \draw (5.5, 2) node {H};
%   \draw (0, 3.3) node (baseline) {baseline};
%   \draw[dashed] (1.7, 3.3) -- (5.9, 3.3);
%   \draw[<->] (5.5, 3.3) -- (5.5, 4.5);
%   \draw (5.9, 3.9) node {h1};
%
%   \draw (6.5, 4) rectangle (9.5, -1);
%   \draw[<->] (9.7, -1) -- (9.7, 4);
%   \draw (10, 1.5) node {H};
%   \draw[<->] (10, 3.3) -- (10, 4);
%   \draw (10.3, 3.7) node {h};
%   \draw[fill=lightgray] (6.6, 1) rectangle (9.4, 3.9);
%   \draw (8, 0)  node[align=center] {\cs{vfill}};
%   \draw (8, 2) node[align=center] {text};
%   \draw[draw=gray] (6.6, 3) -- (9.4, 3);
%   \draw[dashed] (6.2, 3.3) -- (9.8, 3.3);
%   \draw[->, line width=4pt, draw=gray] (8, 3.3) -- (8, 4);
% \end{tikzpicture}
% \caption{Case \meta{nrows}${}> 0$, \meta{vpos} = [t]}
% \label{fig:top-t}
% \end{figure}
%
% \subsubsection*{\meta{vpos}  = [c]}
%
% In this case  the \cs{vbox} contains a \cs{vfill}, the text, and another \cs{vfill}.
% Such a \cs{vbox} has a height 0, i.e. the top of the box is on the
% baseline. Because both boxes have the same size (H), they can be
% aligned by shifting the \cs{vbox} up over h1.  See fig.~\ref{fig:top-c}.
%
% \begin{figure}[htpb]
%   \centering
%\begin{tikzpicture}
%   \draw (2, -0.5) rectangle (5,4.5);
%   \foreach \y in {0,...,4} \draw (2, \y) -- (5, \y);
%   \draw (0, 4.2) node (topstrut) {topstrut};
%   \draw[->] (topstrut) -- (2, 4.2);
%   \draw (0, -0.25) node (botstrut) {botstrut};
%   \draw[->] (botstrut) -- (2, -0.25);
%   \draw[<->] (5.3, -0.5) -- (5.3, 4.5);
%   \draw (5.5, 2) node {H};
%   \draw (0, 3.3) node (baseline) {baseline};
%   \draw[dashed] (1.7, 3.3) -- (5.9, 3.3);
%   \draw[<->] (5.5, 3.3) -- (5.5, 4.5);
%   \draw (5.9, 3.9) node {h1};
%
%   \draw (6.5, 3.3) rectangle (9.5, -1.7);
%   \draw[<->] (9.7, -1.7) -- (9.7, 3.3);
%   \draw (10, 0.8) node {H};
%   \draw[fill=lightgray] (6.6, -0.5) rectangle (9.4, 2.1);
%   \draw (8, 2.7)  node[align=center] {\cs{vfill}};
%   \draw (8, 0.8) node[align=center] {text};
%   \draw (8, -1) node[align=center] {\cs{vfill}};
%   \draw[->, line width=4pt, draw=gray] (8, 3.3) -- (8, 4.5);
%   \draw[draw=gray] (6.5, 4.5) -- (9.5, 4.5);
% \end{tikzpicture}
% \caption{Case \meta{nrows}${}> 0$, \meta{vpos} = [c]}
% \label{fig:top-c}
% \end{figure}
%
% \subsubsection*{\meta{vpos}  = [b]}
%
% Now the  \cs{vbox} contains a \cs{vfill}, followed by the text.
% Because it ends with the text, it gets an additional depth equal to
% the depth of the last line of the text.
% Such a \cs{vbox} has a height 0, i.e. the top of the box is on the
% baseline, but its depth is H + that depth. In other words the baseline
% of the last text line is H below the top.
%
% Because \meta{vpos}  = \texttt{[b]} we want the baseline of the last
% textline to shift to the baseline of the last tabular row. The amount
% of the shift is h1 + h2.  See fig.~\ref{fig:top-b}.
%
% \begin{figure}[htpb]
%   \centering
%\begin{tikzpicture}
%   \draw (2, -0.5) rectangle (5,4.5);
%   \foreach \y in {0,...,4} \draw (2, \y) -- (5, \y);
%   \draw (0, 4.2) node (topstrut) {topstrut};
%   \draw[->] (topstrut) -- (2, 4.2);
%   \draw (0, -0.25) node (botstrut) {botstrut};
%   \draw[->] (botstrut) -- (2, -0.25);
%   \draw[<->] (5.3, -0.5) -- (5.3, 4.5);
%   \draw (5.5, 2) node {H};
%   \draw (0, 3.3) node (baseline) {baseline};
%   \draw[dashed] (1.7, 3.3) -- (5.9, 3.3);
%   \draw[<->] (5.5, 3.3) -- (5.5, 4.5);
%   \draw (5.9, 3.9) node {h1};
%   \draw[<->] (5.5, -0.5) -- (5.5, 0.3);
%   \draw (5.9, -0.1) node {h2};
%
%   \draw (6.5, 3.3) rectangle (9.5, -2.1);
%   \draw[<->] (9.7, -1.7) -- (9.7, 3.3);
%   \draw (10, 0.8) node {H};
%   \draw[fill=lightgray] (6.6, -2) rectangle (9.4, 1);
%   \draw (8, 2)  node[align=center] {{\cs{vfill}}};
%   \draw (8, -0.5) node[align=center] {text};
%   \draw (0, 0.3) node (baseline) {baseline};
%   \draw[dashed] (6.3, -1.7) -- (10, -1.7);
%   \draw (3.5, -1.7) node (baseline) {baseline of last text line};
%   \draw[dashed] (1.7, 0.3) -- (10, 0.3);
%   \draw[->, line width=4pt, draw=gray] (8.5, -1.7) -- (8.5, 0.3);
% \end{tikzpicture}
% \caption{Case \meta{nrows}${}> 0$, \meta{vpos} = [b]}
% \label{fig:top-b}
% \end{figure}
%
% \subsection{Case \meta{nrows}${}<0$}
%
% \meta{nrows}${}<0$ when the multirow is positioned in the last row of
% the multirow block.
%
% \subsubsection*{\meta{vpos}  = [t]}
%
% In this case the \cs{vbox} contains the text followed by a \cs{vfill}.
% Such a \cs{vbox} has a height that is the height of the top line of
% the text. The baseline is aligned with the baseline of the last row.
% Because \meta{vpos}  = [t], we want it to be aligned with the baseline
% of the first row. Therefore it has to be shifted up $\textrm{H} -
% \textrm{h1} - \textrm{h2}$. But because later the height of the box
% will be set to 0, we must also add the current height h. Therefore the
% total shift becomes $\textrm{H} - \textrm{h1} - \textrm{h2} +
% \textrm{h}$.  See fig.~\ref{fig:bot-t}.
%
% \begin{figure}[htpb]
%   \centering
% \begin{tikzpicture}
%   \draw (2, -0.5) rectangle (5,4.5);
%   \foreach \y in {0,...,4} \draw (2, \y) -- (5, \y);
%   \draw (0, 4.2) node (topstrut) {topstrut};
%   \draw[->] (topstrut) -- (2, 4.2);
%   \draw (0, -0.25) node (botstrut) {botstrut};
%   \draw[->] (botstrut) -- (2, -0.25);
%   \draw[<->] (5.3, -0.5) -- (5.3, 4.5);
%   \draw (5.5, 2) node {H};
%   \draw (0, 3.3) node (baseline) {baseline};
%   \draw[dashed] (1.7, 3.3) -- (9, 3.3);
%   \draw[<->] (5.5, 3.3) -- (5.5, 4.5);
%   \draw (5.9, 3.9) node {h1};
%   \draw[<->] (5.5, -0.5) -- (5.5, 0.3);
%   \draw (5.9, -0.1) node {h2};
%   \draw (0, 0.3) node (baseline) {baseline};
%   \draw[dashed] (1.7, 0.3) -- (5.9, 0.3);
%
%   \draw (6.5, 1) rectangle (9.5, -4);
%   \draw[<->] (9.7, -4) -- (9.7, 1);
%   \draw (10, -1.5) node {H};
%   \draw[<->] (10, 0.3) -- (10, 1);
%   \draw (10.3, 0.7) node {h};
%   \draw[fill=lightgray] (6.6, -2) rectangle (9.4, 0.9);
%   \draw (8, -3)  node[align=center] {\cs{vfill}};
%   \draw (8, -1) node[align=center] {text};
%   \draw[draw=gray] (6.6, 0) -- (9.4, 0);
%   \draw[dashed] (6.2, 0.3) -- (10.3, 0.3);
%   \draw[->, line width=4pt, draw=gray] (8, 0.3) -- (8, 3.3);
% \end{tikzpicture}
% \caption{Case \meta{nrows}${} < 0$, \meta{vpos} = [t]}
% \label{fig:bot-t}
% \end{figure}
%
% \subsubsection*{\meta{vpos}  = [c]}
%
% In this case  the \cs{vbox} contains a \cs{vfill}, the text, and another \cs{vfill}.
% Such a \cs{vbox} has a height 0, i.e. the top of the box is on the
% baseline. Because both boxes have the same size (H), they can be
% aligned by shifting the \cs{vbox} up over $\textrm{H} - \textrm{h2}$.
% See fig.~\ref{fig:bot-c}.
%
% \begin{figure}[htpb]
% \begin{tikzpicture}
%   \draw (2, -0.5) rectangle (5,4.5);
%   \foreach \y in {0,...,4} \draw (2, \y) -- (5, \y);
%   \draw (0, 4.2) node (topstrut) {topstrut};
%   \draw[->] (topstrut) -- (2, 4.2);
%   \draw (0, -0.25) node (botstrut) {botstrut};
%   \draw[->] (botstrut) -- (2, -0.25);
%   \draw[<->] (5.3, -0.5) -- (5.3, 4.5);
%   \draw (5.5, 2) node {H};
%   \draw (0, 3.3) node {baseline};
%   \draw[dashed] (1.7, 3.3) -- (5.8, 3.3);
%   \draw[<->] (5.5, 3.3) -- (5.5, 4.5);
%   \draw (5.9, 3.9) node {h1};
%   \draw[<->] (5.5, -0.5) -- (5.5, 0.3);
%   \draw (5.9, -0.1) node {h2};
%   \draw (0, 0.3) node {baseline};
%   \draw[dashed] (1.7, 0.3) -- (5.9, 0.3);
%
%   \draw (6.5, 0.3) rectangle (9.5, -4.7);
%   \draw[<->] (9.7, -4.7) -- (9.7, 0.3);
%   \draw (10, -3.2) node {H};
%   \draw[fill=lightgray] (6.6, -3.5) rectangle (9.4, -0.9);
%   \draw (8, -0.3)  node[align=center] {\cs{vfill}};
%   \draw (8, -2.2) node[align=center] {text};
%   \draw (8, -4) node[align=center] {\cs{vfill}};
%   \draw[->, line width=4pt, draw=gray] (8, 0.3) -- (8, 4.5);
%   \draw[draw=gray] (6.5, 4.5) -- (9.5, 4.5);
% \end{tikzpicture}
% \caption{Case \meta{nrows}${} < 0$, \meta{vpos} = [c]}
% \label{fig:bot-c}
% \end{figure}
%
% \subsubsection*{\meta{vpos}  = [b]}
%
% The  \cs{vbox} contains a \cs{vfill}, followed by the text.
% Because it ends with the text, it gets an additional depth equal to
% the depth of the last line of the text.
% Such a \cs{vbox} has a height 0, i.e. the top of the box is on the
% baseline, but its depth is H + that depth. In other words the baseline
% of the last text line is H below the top.
%
% Because \meta{vpos}  = \texttt{[b]} we want the baseline of the last
% textline to shift to the baseline of the last tabular row. The amount
% of the shift is H.  See fig.~\ref{fig:bot-b}.
%
% \begin{figure}[htpb]
% \begin{tikzpicture}
%   \draw (2, -0.5) rectangle (5,4.5);
%   \foreach \y in {0,...,4} \draw (2, \y) -- (5, \y);
%   \draw (0, 4.2) node (topstrut) {topstrut};
%   \draw[->] (topstrut) -- (2, 4.2);
%   \draw (0, -0.25) node (botstrut) {botstrut};
%   \draw[->] (botstrut) -- (2, -0.25);
%   \draw[<->] (5.3, -0.5) -- (5.3, 4.5);
%   \draw (5.5, 2) node {H};
%   \draw (0, 3.3) node {baseline};
%   \draw[dashed] (1.7, 3.3) -- (5.8, 3.3);
%   \draw[<->] (5.5, 3.3) -- (5.5, 4.5);
%   \draw (5.9, 3.9) node {h1};
%   \draw[<->] (5.5, -0.5) -- (5.5, 0.3);
%   \draw (5.9, -0.1) node {h2};
%   \draw (0, 0.3) node {baseline};
%   \draw[dashed] (1.7, 0.3) -- (5.9, 0.3);
%
%   \draw (6.5, 0.3) rectangle (9.5, -5.1);
%   \draw[<->] (9.7, -4.7) -- (9.7, 0.3);
%   \draw (10, -3.2) node {H};
%   \draw[fill=lightgray] (6.6, -5) rectangle (9.4, -2);
%   \draw (8, -1)  node[align=center] {{vfill}};
%   \draw (8, -3.5) node[align=center] {text};
%   \draw (0, 0.3) node (baseline) {baseline};
%   \draw[dashed] (1.7, 0.3) -- (10, 0.3);
%   \draw[dashed] (6.3, -4.7) -- (10, -4.7);
%   \draw (3.5, -4.7) node {baseline of last text line};
%   \draw[->, line width=4pt, draw=gray] (8.5, -4.7) -- (8.5, 0.3);
% \end{tikzpicture}
% \caption{Case \meta{nrows}${} < 0$, \meta{vpos} = [b]}
% \label{fig:bot-b}
% \end{figure}
%
% \subsection{Overview}
% \label{sec:overview}
%
% \begin{tabular}{| l | l | l |}
%     \hline
%   \meta{vpos} & \meta{nrows} > 0  & \meta{nrows} < 0 \\
%   \hline
%   \texttt{[t]}  & h            &  $\textrm{H} - \textrm{h1} -\textrm{h2} + \textrm{h}$  \\
%   \texttt{[c]}  & h1          &  $\textrm{H} - \textrm{h2}$ \\
%   \texttt{[b]}  & h1 + h2 &   H \\
%   \hline
%                    &  x          &   $\textrm{H} - \textrm{h1} -\textrm{h2} + x$ \\
% \hline
%   \end{tabular}
%
% \Finale
\endinput
